O objetivo do robô ROSA é de otimizar o processo de inserção e remoção de
stoplogs, operações frequentes para atividades de manutenção na usina. A
importância dessa otimização se dá à medida que o tempo de parada de máquinas
pode diminuir consideravelmente com o monitoramento do processo, assim como
acidentes de operação. Através deste monitoramento é possível identificar
problemas mecânicos como o emperramento nos trilhos e falhas de pesca de
stoplogs, assim como problemas oriundos do contexto ambiental da usina, como o
grande volume de detritos acumulados nas soleiras que impedem o perfeito
assentamento dos stoplogs.

Ao abordar o problema foi entendido que o processo se dá de forma puramente
mecânica, onde o operador obtém feedback através de barulhos do motor e mudança
na tensão dos cabos que seguram os stoplogs. Desta forma, sensores indutivos,
inclinômetro, profundímetro foram estruturados na viga pescadora para monitorar
seu funcionamento e dar ao operador um contexto atualizado do seu ambiente de
operação. Assim o aplicativo ROSA, visualizado em um tablet dentro da cabine do
operador, permite um feedback visual e sonoro das operações realizadas.

Desta forma o aplicativo facilita o processo de remoção e inserção de stoplog a
medida que aponta erros comuns na operação que antes não podiam ser determinados com
precisão. Ao antever possíveis problemas, é possível mitigar alguns processos e
dar prosseguimento a manutenção, aumentando a disponibilidade das máquinas e,
consequentemente, fornecendo energia para o sistema elétrico brasileiro
