\section{Relatório de Viagem Jirau - Junho 2014}
Data: 02 a 06 de Junho de 2014.
Equipe: Alana Monteiro, Eduardo Elael, Gabriel Alcântara, Julia Campana e Renan
Salles.

O objetivo da viagem foi realizar os primeiros testes de campo do robô e
alinhamento administrativo entre equipes.

Foram realizados testes de alguns dos dispositivos que compõem o
robô ROSA, entre eles: eletrônica embarcada, sonar e sensores indutivos. Também
foi de grande importância o reconhecimento do material utilizado na usina,
estrutura das vigas pescadoras e o processo de trabalho dos operadores de vigas
de stoplog.

Como resultado dos testes foi descoberto que a estrutura mecânica desenvolvida
se mostrou inadequada uma vez que a mesma foi projetada de forma simétrica e a
viga apresenta uma barra rígida entre as garras, não simétrica, impedindo assim
a conexão com do sensor indutivo. Não houveram falso positivos ou danos nos
sensores e demais equipamentos.

\begin{figure}[h!]
\centering
  \includegraphics[width=1\linewidth]{Fotos/JirauJunho2014/1.jpg}
  \caption{Sonar acoplado à viga pescadora para a execução de testes.}
  \label{nov20131}
\end{figure}

\begin{figure}[h!]
  \centering
  \includegraphics[width=1\linewidth]{Fotos/JirauJunho2014/7.jpg}
  \caption{Equipe analiza dados coletados em campo.}
  \label{nov20132}
\end{figure}

\begin{figure}[h!]
  \centering
  \includegraphics[width=1\linewidth]{Fotos/JirauJunho2014/8.jpg}
  \caption{Engenheiros da equipe ajustam dispositivos para realização de
  testes.}
  \label{nov20133}
\end{figure}

\begin{figure}[h!]
  \centering
  \includegraphics[width=1\linewidth]{Fotos/JirauJunho2014/10.jpg}
  \caption{Eletrônica embarcada presa a viga pesacadora para a execução de
  testes.}
  \label{nov20134}
\end{figure}