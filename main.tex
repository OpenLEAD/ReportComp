\documentclass[a4paper,11pt,oneside,brazilian,
draft=false]{report}%openany,version=last

\usepackage{float}
\usepackage{rotating}
\usepackage{graphicx}
\usepackage{mybook}
\usepackage{fancyhdr}
\usepackage{pdflscape}
\usepackage{pdfpages}
\usepackage[toc,page]{appendix}
\usepackage{titlesec}
\usepackage{xcolor}
\usepackage{framed}
\usepackage{titletoc}
\usepackage{longtable}
\usepackage[brazilian]{babel} 
\usepackage{movie15}
\usepackage{hyperref}

\newcommand*{\fancychapterstyle}{%
  \titleformat{\chapter} % command
	[block] % shape
	{\bfseries\Large\itshape} % format
	{	\textsc{\LARGE Compilação de relatórios técnicos} 
		\newline \newline  
		\textsc{\Large Capítulo \ \thechapter}
	} % label
	{0.5ex} % sep
	{
    	\rule{\textwidth}{2pt}
    	\vspace{1ex}
    	\centering
    	\huge \textnormal
	} % before-code
	[
		\vspace{-1ex}%
		\rule{\textwidth}{2pt}
		\vfill \includegraphics{logo/rosa_logo.png}
	] % after-code
}

\newcommand*{\standardchapterstyle}{%
  \titleformat{\chapter}[display]
  {\normalfont\huge\bfseries}{\chaptertitlename\ \thechapter}{20pt}{\Huge}
  \titlespacing*{\chapter}{0pt}{50pt}{40pt}
}
 
\pdfcompresslevel = 9
\oddsidemargin = 31pt
\topmargin = 20pt
\headheight = 12pt
\headsep = 25pt
\textheight = 592pt
\textwidth = 390pt
\marginparsep = 10pt
\marginparwidth = 35pt
\footskip = 30pt
\pagenumbering{gobble}

\begin{document} 
 
 
\includepdf{capas/capa_main.pdf}
%%******************************************************************************
%%
%% frontpage.tex
%%
%%******************************************************************************
%%
%% Title......: ROSA - Stoplog Inspection
%%
%% Author.....: GSCAR-DFKI
%%
%% Started....: Nov 2013
%%
%% Emails.....: renan028@gmail.com
%%
%% Address....: Universidade Federal do Rio de Janeiro
%%              Caixa Postal 68.504, CEP: 21.945-970
%%              Rio de Janeiro, RJ - Brasil.
%%
%%******************************************************************************


%%******************************************************************************
%% FRONT PAGE
%%******************************************************************************




%%==============================================================================
%% FRONT PAGE CONTENTS
%%==============================================================================
\thispagestyle{empty}

%% Restart page counter.
\setpagecounter{0}

%% Disable page anchor to avoid multiple page number definition warnings.
\hypersetup{pageanchor=false}

\vspace{4cm}

 \textcolor{gray}{Execução:} \\
\\
\begin{minipage}{\textwidth}
	\centering
       
	\includegraphics[width=0.3\textwidth]{logo/lead-logo}
	
\end{minipage}

\vspace{2cm}

\textcolor{gray}{Financiamento: } \\ 
\\
\begin{minipage}{\textwidth}
	\centering
	
	\includegraphics[width=0.3\textwidth]{logo/esbr-logo}
	\includegraphics[width=0.3\textwidth]{logo/aneel-logo}

	
\end{minipage}

\vspace{4cm}

\begin{table}[ht!]
	\centering
	\begin{tabular}{r l|l p{12cm} }
		\textcolor{gray}{Projeto} &&& \textbf{\Large ROSA}\\
			&&& \textbf{Robô para Operação de Stoplogs Alagados}\\
			&&& \\
		\textcolor{gray}{Título} &&& \textbf{Compilação de relatórios técnicos}\\
		\textcolor{gray}{PD} &&& 6631-0002/2013 \\
		\textcolor{gray}{Contrato} &&& Jirau 151/13\\
		\textcolor{gray}{Coordenador} &&& Ramon Romankevicius Costa \\
		\textcolor{gray}{Gerente} &&& Breno Bellinati de Carvalho \\
		\textcolor{gray}{Data:} &&& \today \\
	\end{tabular}
\end{table}


\cleardoublepage

%%==============================================================================
%% AUTHORS AND VERSION PAGE
%%==============================================================================

\thispagestyle{empty}

%% Restart page counter.
\setpagecounter{0}

\begin{center}
  %% Version section. --------------------------------------------------------

  
 \vfill
  %% Project section. --------------------------------------------------------


  %% Authors section. --------------------------------------------------------

  {\LARGE Versão}
  \vspace{0.50cm}



  \begin{center}
    \begin{tabular}{| l | l | l |}
    \hline
   	 Versão 					& Autor			 & Descrição 	 \\ \hline
   	 1.0 		& Renan  	& Implementação Inicial\\			
    
    \hline 
    \end{tabular}
\end{center}






\end{center}

\newpage

%% Enable page anchor again.
\hypersetup{pageanchor=true}



\tableofcontents

\fancychapterstyle

\chapter{Resumo de projeto}
\section{Introdução}
Este é o Relatório Final do projeto intitulado ROBÔ PARA OPERAÇÃO DE STOPLOGS
ALAGADOS – ROSA. Este relatório tem o objetivo de cumprir as determinações da
ANEEL - Agência Nacional de Energia Elétrica e também de ser um resumo
executivo do projeto.

O projeto foi desenvolvido entre a Energia Sustentável do Brasil S.A. e o
Laboratório de Controle e Automação - LEAD, Engenharia de Aplicação e
Desenvolvimento, dentro do Programa de Pesquisa e Desenvolvimento Tecnológico
dessa empresa, no ano base de 2013, conforme regras vigentes da ANEEL, tendo
duração 18 meses.

O presente projeto de pesquisa e desenvolvimento visou desenvolver um protótipo
de sistema robótico visando o monitoramento da operação de inserção e remoção
de stoplogs nas usinas hidrelétricas.

Este relatório está dividido conforme segue.

\section{Revisão Bibiliográfica}

No estudo de Lewin \cite{jack} sobre as propriedades físicas e mecânicas de 
\textit{stoplogs}, o autor afirma, em seu livro, que o
\textit{stoplog} ideal possuiria um sensor de contato e uma válvula reguladora
de pressão hidrostática, o que resolveria a maioria dos problemas apresentados
em operações de instalação e remoção de \textit{stoplogs}. Entretanto, em Group
\cite{pinc}, Inglaterra, foi proposta uma norma técnica para a produção de
\textit{stoplogs} sem o sensoriamento previamente citado, devido aos altos custos de implementação e manutenção. Em
2006, Hatch, uma consultoria de engenharia e desevolvimento,
desenvolveu um sistema composto por uma viga pescadora instrumentada com sensores de
proximidade e servomotores elétricos submersíveis para atuar, idependentemente,
em cada garra. Em 2007, o sistema foi instalado na represa do Lago Ivanhoe,
reduzindo consideravelmente a necessidade de intervenção manual e aumentando,
assim, a segurança da operação (Hatch,2009)\cite{hatch}.

Em uma aplicação semelhante, a Atlas Polar automatizou o processo
de instalação e remoção de stoplogs. O sistema consiste na utilização de um
sensor indutivo para a leitura do contato com o \textit{stoplog} e um sensor de força, do tipo
\textit{strain gauge}, para indicar a liberação ou içamento de
\textit{stoplogs} (Polar,2015)\cite{atlas}.

As soluções presentes na literatura não são viáveis comercialmente
para o contexto da usina de Jirau. As propostas requerem a aquisição de todo o
sistema composto por pórtico rolante e viga pescadora para a integração dos
sensores. Não foi encontrado, também, nenhum trabalho anterior referenciando a
utilização de sonares para a inspeção do vão com os trilhos guia e da soleira da
entrada e saída da turbina.


\section{Justificativa do desenvolvimento do projeto}
Visando um aumento na segurança e otimização das operações realizadas na usina 
hidrelétrica de Jirau (3,750 MW Brasil), a Energia sustentável do Brasil (ESBR)
estabelceu uma parceria com a Universidade Federal do Rio de Janeiro para o desenvolvimento de um projeto de 
inovação: o robô submarino ROSA para monitoramento e inspeção das operações com
\textit{stoplogs}. O projeto ROSA teve início em Novembro de 2013 e término em
Fevereiro de 2015, totalizando um investimento de R\$ 2.8 milhões.

\textit{Stoplogs} são barreiras metálicas modulares que são empilhadas na
entrada e sáida de cada tubina, permitindo o vedamento e a drenagem da
região para manutenção. Em Jirau, cada \textit{stoplog} possui massa de duas
toneladas e é necessário empilhar oito em cada extremidade da turbina para o seu
vedamento.

Um operador instala e remove os \textit{stoplogs}, utilizando um pórtico
rolante e uma viga pescadora. Quando a viga pescadora está submersa, a tensão
do cabo que a sustenta é a única informação disponível ao operador. Essas
operações às cegas resulta, ocasionalmente, em falhas causadas por
influências mecânicas ou ambientais. Alguns exemplos são:

\begin{itemize}
\item \textit{Acoplamento incompleto}. Viga pescadora acopla apenas um dos
olhais do \textit{stoplog}, resultando em uma movimentação inclinada e danificando a estrutura;
\item \textit{Detritos e sedimentos}. Acúmulo de detritos na soleira,
acarretando em vedação imperfeita e, consequentemente, impossibilitando uma
drenagem efetiva.
\end{itemize}

Falhas na operação podem ser resolvidas somente por meio da intervenção de
mergulhadores. Entretanto, o fluxo das turbinas adjacentes em operação e o
turbilhamento do rio podem ocasionar em acidentes, tornando essa medida uma 
tarefa de alto risco à vida dos mergulhadores.


\section{Aspecto inovador do projeto}

O desenvolvimento de um sistema robótico submarino capaz de mapear e monitorar
operações com \textit{stoplogs}. O ganho econômico, por meio da
redução das paradas na produção, e o aumento da segurança da operação,
pela redução da necessidade da intervenção de mergulhadores, motivam o
desenvolvimento da solução.

\section{Descrição da inovação - Originalidade}

O robô de monitoramento e mapeamento submarino, ROSA, é projetado para evitar a
maioria dos problemas relacionados às operações com \textit{stoplogs}. O sistema
submarino realiza o monitoramento, em tempo real, processando e apresentando os
dados diretamente em um tablet instalado na cabine do operador. Os dados do
sistema auxiliam o operador no processo de tomada de decisão e alarmes indicam
falhas na operação. É válido destacar que não há em nenhuma outra usina um robô
que realize tais tarefas, sendo esse então um projeto inovador e de extrema
relevância. 

O sistema consiste em uma eletrônica submarina embarcada, desenvolvida
exclusivamente para o projeto, que se comunica com os sensores integrados à
viga. As medidas realizadas pelo sensor são transmitidas para a
eletrônica em terra através de um umbilical e, após o processamento, são
transmitidas via Wi-Fi para o tablet com o operador.

O conjunto de sensores responsáveis pelo monitoramento é constituído por
sensores de inclinação, profundidade e proximidade. Cada etapa do processo de
pesca e remoção dos \textit{stoplogs} é, então, monitorado.

O sistema de mapeamento é constituído por uma eletrônica submarina e um sensor
sonar acoplado a uma unidade pan e tilt. A informação aquisitada pelo sonar é
apresentada ao operador, sobreposta a um modelo 3D da estrutura submersa, possibilitando, a
visualização e identificação intuitiva dos sedimentos e detritos acumulados.
As ondas sonoras emitidas e recebidas pelo sonar são processadas 
probabilisticamente e, de acordo com a potência e tempo de resposta
do eco recebido, essas informações são inseridas em um mapa 3D de ocupância
probabilística. Por meio de um filtro de partículas, o resultado mais provável
é estimado.

O projeto acrescenta uma inovação para o processo atual, no qual a operação é
realizada sem monitoramento. A solução é aplicável a qualquer usina que utilize
um sistema de \textit{stoplogs} baseado em pórtico rolante e viga pescadora.


\section{Aplicabilidade}

O resultado final do projeto foi um protótipo consistindo de: uma unidade de
superfície à prova d'água, certificada com IP68 (nível de classes de proteção,
padrão internacional definido pela norma IEC 60529 para classificar e avaliar
o grau de proteção de produtos fornecidos contra intrusão), capaz de suportar
umidade e altas temperaturas; e uma unidade submarina, certificada com IP69K,
capaz de suportar até 100 bar. Atualmente, o sistema está instalado e sendo
utilizado na Usina Jirau.

O protótipo foi demonstrado para os representantes das Usina Jirau e Santo
Antônio em uma operação real com \textit{stoplogs}. A tarefa de monitoramento
foi realizada com sucesso, indicando todos os passos de uma operação e emitindo
alarmes para as falhas detectadas. O mapeamento também foi bem sucedido,
detectando detritos maiores que 20cm. 

ESBR e Santo Antônio demostraram interesse
na continuação de projetos de inovação na mesma frente de pesquisa. O
equipamento de monitoração deve ter seu custo de implementação otimizado e
deverá ser instalado permanentemente em cada viga pescadora, totalizando 21
unidades em ambas as usinas. A unidade de mapeamento deve ter
sua capacidade estendida para poder mapear os danos causados pela cavitação nas
paredes de concreto.

\section{Relevância}

O ganho econômico gerado pelo sistema ROSA pode ser estimado pela quantidade de
falhas operacionais com \textit{stoplogs} que podem ser evitadas com a
utilização diária do sistema na Usina Jirau. Durante um período de 2 anos,
esse tipo de falha resultou em uma perda de disponibilidade de produção (Aumento
da Energia Disponível) de 37 dias. Considerando que cada turbina em Jirau produz
75 MWh e o preço da energia no mercado brasileiro ser aproximadamente R\$186,00
por MWh, a prevenção dessas falhas resultariam em uma economia de
R\$12.358.238,04, no período.

O ganho de segurança do trabalho com a utilização do robô ROSA pode ser estimado
pela redução do número de intervenções necessárias com a utilização de
mergulhadores. O risco desse tipo de intervenção é classificado pela HSE
(\textit{Health Security and Environment} - Saúde, Segurança e Ambiente) como
elevado, uma vez que representa um risco real de perda de vidas humanas. Desde
2014, são registrados incidentes em usinas hidrelétricas relacionados a
intervenção envolvendo a utilização de mergulhadores. 

Nos últimos dois anos, os problemas relacionados a \textit{stoplogs}
resultaram em 25 dias de intervenções com mergulhadores. A perda de
produção é relacionada ao tempo necessário em mobilizar os mergulhadores,
colocá-los na água e, finalmente, resolver o problema em si. O robô ROSA é um
sistema de monitoramento e inspeção que tem como objetivo auxiliar o
operador, a fim de reduzir a ocorrência de falhas e evitar completamente ações
que resultem em um agravamento das condições de operação após uma falha ter
ocorrido. Considerando a lista de ocorrências envolvendo \textit{stoplogs}
abaixo, a utilização do robô ROSA resultaria em uma redução de 75\% da
exposição ao risco de acidentes.

Seguem algumas falhas registradas:S
\begin{enumerate}
\item Não ensecamento de Unidade Geradora (UG) devido a detritos na soleira e
guia lateral:
\begin{itemize}
\item UG 01, outubro/2014, 4 dias
\item UG 03, outubro/2014, 4 dias
\item UG 05, setembro/2014, 4 dias
\item UG 32, setembro/2014. 4 dias	
\end{itemize}
OBS: Por consequência, todas as paradas de máquina, a partir de
outubro de 2014, têm tido, necessariamente, mergulhos
preventivos. 
\item Queda de painel devido à pesca equivocada:
\begin{itemize}
\item UG 32 (antes do enchimento), fevereiro/2013; 10 dias (recuperação
soleira e guia lateral)
\item Vertedouro Principal vão 5, dezembro 2013. 4 dias (tentativa de
localização com sonar pois o mesmo saiu do vão)
\end{itemize}
\item Excesso de cabo do pórtico se emaranha no painel e se rompe na
tentativa de içamento. 7 dias de atraso para retirar e 90 dias para chegada de
novo cabo. Levando em conta a resolução de todos os problemas, o pórtico fica limitado por mais de 100 dias.
\end{enumerate}

O projeto, apesar de curta duração (16 meses), resultou em três dissertações de 
mestrado, uma publicação técnica submetida ao CITENEL 2015, um artigo técnico 
em inglês submetido ao prêmio GDF Suez de inovação, e seis publicações na
imprensa.

%TODO Publicaçõe de imprensa
 
     
\section{Razoabilidade dos Custos}
O projeto ROSA custou R\$2.880.000,00 em sua fase de cadeia de pesquisa
aplicada. Ao final do projeto, constatou-se a necessidade de onze unidades 
do robô ROSA, cada uma instalada permanentemente em uma viga pescadora. O custo
do ROSA, após a otimização dos componentes e fabricação, é esperado ser de
R\$35.000,00 por unidade. Logo, o custo de implantação da solução seria de R\$385.000,00,
totalizando um custo total de R\$3.265.000,00 entre conceito e implantação.

O ganho econômico (aumento de energia disponível), que o ROSA teria
proporcionado se em operação nos últimos 2 anos, em Jirau, é de R\$
12.358.238,04, como detalhado na seção anterior. Portanto, o projeto teria uma
expectativa de retorno de investimento de 6 meses.

\section{Transferência e difusão tecnológica dos resultados do projeto}
Um workshop de transferência de conhecimento foi realizado na UHE Jirau com os
Engenheiros de operação. Além disso, foi provido treinamento para manutenção e
uso do protótipo, incluindo manual de uso. Ao total, um engenheiro de operação foi 
treinado na manutenção do software, dois técnicos instrumentistas foram treinados 
na instalação e uso do equipamento, e dois operadores de pórtico rolante no uso 
do equipamento.

\section{Metodologia}

Após a análise e compreensão do problema de inserção e remoção de
\textit{stoplogs}, foi feita uma pesquisa bibliográfica e \textit{brainstorm}
com o objetivo de alcançar um conceito sólido de solução do problema. A partir
do resultado dessa pesquisa, foi desenvolvido um conceito base de solução robótica. 
Baseado neste conceito, foram realizadas pesquisas de tecnologias e de fornecedores 
de forma recursiva e convergente com relação aos resultados. Isto é, com base
nas pesquisas de soluções tecnológicas possíveis, buscam-se fornecedores
compatíveis e, com o resultado e informação dos produtos dos fornecedores
encontrados, faz-se novamente uma pesquisa de tecnologia, agora mais
aprofundada, e assim sucessivamente, até encontrar-se um resultado final
satisfatório.

O escopo inicial de solução é, então, atualizado e detalhado de acordo com o 
resultado desta pesquisa, resultando no design do robô a ser construído no projeto. 
Após a especificação do design, os componentes necessários para construção são
adquiridos e manufaturados. Os sensores e componentes eletrônicos são testados
em campo para averiguar que funcionam como especificado e planejado no
projeto. Em paralelo, o software e interfaces necessários para o projeto são 
detalhados e implementados. 

Após a etapa de detalhamento, o projeto entra na etapa de execução, na qual
ocorre a integração dos diversos componentes que constituem o robô.
O software necessário para o processamento dos dados é desenvolvido e o
sistema, como um conjunto, é testado em campo. 

A viagem de campo fornece informações sobre as falhas, limitações e melhorias 
necessárias ao sistema. O projeto entra, então, na etapa de encerramento. Nesta 
etapa, o sistema é aprimorado e testado recursivamente e o conhecimento de uso e 
operação do mesmo é transferido. Por fim, a última etapa do projeto consiste na
entrega do equipamento, manual de uso e documentação de encerramento do projeto.

\section{Objetivos geral e específico}

O objetivo do projeto ROSA é reduzir o tempo de parada de máquina em razão da
vedação do circuito hidráulico. Uma solução para monitoramento de stoplog que
possibilite identificar problemas como o emperramento no trilho, falhas na
pesca do stoplog e mal assentamento, em geral causado pelo acúmulo de silte
(fragmento de mineral ou rocha menor do que areia fina e maior do que argila).

Para atingir esse objetivo o sistema proposto deveria ser capaz de realizar as
seguintes tarefas:

\begin{itemize}
	\item Monitorar os engates das garras da viga pescadora nos olhais do stoplog;
	\item Monitorar a inclinação da viga pescadora;
	\item Monitorar a profundidade em que se encontra o stoplog;
	\item Escanear a soleira com sonar para identificar a presença de silte e 
	detritos; 
	\item Montar uma interface para leitura amigável dos dados e interação com os
	sistema.
\end{itemize}

\section{Planejamento}
\includepdf{PDFs/Plano_Trabalho.pdf}

\definecolor{Gray}{gray}{0.9}

\begin{longtable}{ | l | p{4cm} | l | l | p{5cm} | }
\hline
    \rowcolor{Gray}
	AP* & Descrição & Início & Dur &  \  \\ \hline
	 & Projeto Basico & 1 & 4 & \ \\ \hline
	1 & Viagem Inaugural & 1 & 1 &  Assinatura do termo inaugural do projeto e
	análise em campo da problemática \\ \hline 
	2 & Requisito & 1 & 2 & Neste pacote de trabalho os requisitos do sistema serao
	especificados através de reunioes com a ESBR e análise em campo da operacao.    \\ \hline 
	3 & Contextualização & 1 & 2 & Levantamento das tecnologias existentes para monitoramento e mapeamento de operacoes de Stoplog \\ \hline
	4 & Definição do conceito & 2 & 3 & Definição de uma solução de um robô capaz
	 de operar no ambiente e realizar tarefas de coating   \\ \hline 
	5 & Análise do Usuário & 3 & 2 & Análise dos operadores de pórtico rolante que
	 serao os usuários finais do sistema de monitoramento.  \\ \hline 
	6 & Estudo dos \newline fornecedores & 4 & 1 & Definição dos fornecedores
	de equipamento \\ \hline 
	\rowcolor{Gray}
	  & Detalhamento & 5 & 4 &  \  \\ \hline
	7 & Design do Sistema & 5 & 3 & Processo de definicao da arquitetura,
	componentes,,ódulos e interfaces que satisfazem os requisitos do sistema. \\
	\hline 8 & Manufatura \newline e Aquisicao & 5 & 4 & Compra e construção dos
	componentes definidos durante a fase de design do sistema. O resultado será as partes que integradas formarão o robô.  \\ \hline 9 & Octomap & 5 & 4
	& Integracao da biblioteca de acumulacao volumétrica ao software de robótica
	do projeto \\ \hline 10 & Protótipo de Interface & 5 & 4 & Desenvolvimento de
	tela sinterativas simples, concentrando apenas no desenvolvimento da parte
	visual da interface. \\ \hline 11 & Drivers & 6 & 4 & Protocolos de
	comunicacao entre o computador embarcado e os sensores comprados \\ \hline 12
	& Viajem Jirau - Teste de Sensor & 8 & 1 & Viagem a Jirau para teste simples
	dos sensores adiquiridos  \\ \hline
	\rowcolor{Gray}
	   & Execucao & 9 & 6 &  \\ \hline
	13 & Modelagem do Sonar Profiling & 9 & 3 & Modelagem da resposta da potência
	do ECO do sonar no mapa de acumulacao volumétrica Octomap.  \\ \hline 
	14 & Reconstrucao 3D Simples & 9 & 1 & Vizualisacao 3D de uma acumulacao
	volumétrica simples (cubo)  em Octomap  \\ \hline 
	15 & Integracao Eletrônica & 9 & 4 & Montagem da eletrônica
	embarcada \\ \hline 
	16 & Interface de Usuario & 9 & 4 & Implementacao da interface de
	usuário (GUI) do robo, aplicativo para applet Android \\ \hline 
	17 & Localizacao & 10 & 2 & Algoritmo de localizacao das medicoes do
	sonar com relacao ao vao do Stoplog \\ \hline 
	18 & Modelagem de Sonar Imaging & 12 & 3 & Modelagem da resposta da
	potência do ECO de sonar tipo imaging no mapa de acumulacao volumétrica Octomap. \\ \hline 
	19 & Filtro de Partícula & 12 & 3 & Implementação do filtro
	probabilítico  para filtrar as diverss medidas de sonares do meio com relacao a sua distribuicao de probabilidade. \\ \hline 
	20 & Integracao Sistema & 12 & 2 & Integração electro-mecanica e de
	 software do robô ROSA \\ \hline 
	21 & Viajem Jirau - Teste de Sistema & 13 & 1 & Testes do protótipo em
	Jirau \\ \hline
	\rowcolor{Gray}
	 & Encerramento & 15 & 2 & \  \\ \hline
	22 & Tuning & 15 & 2 & Ajustes finos dos resultados da pesquisa
	baseado nos testes de campo \\ \hline 
	23 & Viajem Jirau - Teste de Sistema & 15 & 1 & Viagem a Jirau para testes
	do sistema \\ \hline 
	24 & Workshop Troca Conhecimento & 16 & 1 & Viagem a Jirau para treinar a
	equipe na utilizacao do protótipo e realizacao do workshop de transferencia de conhecimento.  \\ \hline 
	\rowcolor{Gray}
	   & Marcos &  &  &  \  \\ \hline
	M1 & Projeto Básico & 4 & 1 &  \  \\ \hline
	M2 & Design do Sistema & 9 & 1 & \  \\ \hline
	M3 & Demonstracao do Sistema & 13 & 1   & \  \\ \hline
	M4 & Relatório \newline Encerramento & 16 & 1   & \  \\ \hline
\end{longtable}



% \section{Financeiro}
% \begin{center}
%   \begin{tabular}{ | c | c | c | c | p{4.9cm} | }
%     \hline
%         \textbf{Rúbrica} & \textbf{Valor Previsto (R\$)} & \textbf{Valor Realizado (R\$)} & \textbf{Devio (\%)} & \textbf{Justificativa para os desvios positivos} \\ \hline
%     RH & 1.207.409,64 & 975.288,71 & 95\%  &  \\ \hline
%     MC & 2500,00      & 11.314,90  & 453\% &  Custo foi subestimado na estimativa  inicial dos componentes para construção do protótipo \\ \hline
%     MP & 1.972.875,00 & 987.282,99 & 50\%  &  \\ \hline
%     ST & 700.000,00   & 433.904,15 & 62\%  &  \\ \hline
%     VD & 158.160,00   & 95.871,00  & 61\%  &  \\ \hline
%     OU & 503.273,14   & 376.338,75 & 75\%  &  \\ 
%     \hline
%   \end{tabular}
% \end{center} 
 
\section{Justificativa das alterações das informações relacionadas ao projeto
base do XML inicial e a composição do XML final}
\newcolumntype{C}[1]{>{\centering}m{#1}}

A Energia Sustentavel do Brasil (ESBR) optou por financiar como contrapartida no
projeto as rubricas que foram orçadas incialmente como custo da empresa
para realização do projeto, tais como Recursos Humanos (RH), Viagens Diárias
(VD), Material de Consumo (MC), Outros (OU), portanto todos os custos das
rubricas foram absorvidos pela empresa sendo retirados da conta da empresa e
não dos recursos do P&D. Portanto, todos os custos que a equipe da empresa teve
com o projeto foram custeadas pela ESBR.

Vale ressaltar ainda que, no arquivo do xml original constava como equipe da
executora do projeto a Fundação COPPETEC/UFRJ:

\begin{itemize}
\item Ramon Romankevicius Costa (CO);
\item Fernando Cesar Lizarralde (PE);
\item Sylvain Joyeux (PE);
\item Gustavo Medeiros Freitas (PE);
\item Thiago Toledo (PE);
\item Julia Campana (PE);
\item Rodrigo Fonseca Carneiro (PE);
\item Anderson Patury Sangreman (PE);
\item Sabrina Kunst (AT);
\item Joyce Mergulhão de Araújo (AT);

\end{itemize}


Porém, esta equipe foi alterada no decorrer do projeto, tendo em vista que 
o arquivo eletrônico XML do Projeto de P&D foi encaminhado para ANEEL em 27 de março de 2013 por meio do Sistema de Gestão de P&D/ANEEL, e o arquivo
eletrônico XML de início de execução do Projeto foi encaminhado para ANEEL em
08 de outubro de 2013 por meio do Sistema de Gestão de P&D/ANEEL. A
assinatura do Contrato entre a Energia Sustentável do Brasil S.A. (ESBR) e a
Fundação Coordenação de Projetos, Pesquisa e Estudos Tecnológicos –
COPPETEC/UFRJ foi em 08 de outubro de 2013. Porém, durante este intervalo
de tempo a equipe proposta originalmente no arquivo do Projeto encaminhado em
27 de março de 2013 a ANEEL teve alteração, pois alguns dos pesquisadores que
haviam sido selecionados para compor a equipe do projeto receberam propostas
de trabalho, portanto, houve a necessidade de uma nova seleção, mas somente
após a assinatura do contrato foi que a COPPETEC autorizou a contratação dos
demais integrantes para compor a equipe a qual contava somente com o
coordenador Ramon Costa e a pesquisadora Julia Campana. Devido ao novo
cenário conforme supracitado foi aberto um processo seletivo, coordenado pelo
Prof. Ramon Costa, mas que devido à dificuldade e escassez de disponibilidade
de interessados e, visando não comprometer o andamento e quiçá o sucesso da
dos estudos da pesquisa do referido projeto, o coordenador Ramon abriu o convite
para alguns alunos formados do curso de Engenharia e Controle e Automação da
Poli/UFRJ os quais haviam se inscrito para o mestrado no Programa de
Engenharia Elétrica da COPPE, sendo selecionados os pesquisadores para o
projeto: Renan Salles de Freitas, Gabriel Alcântara Costa Silva e Eduardo Elael
de Mello.
Posteriormente, durante a execução do projeto foi detectado a necessidade de um
suporte para a demanda burocrática do processo administrativo para o projeto que
até então não estava no escopo da proposta inicial encaminhada a ANEEL.
Portanto, um outro processo seletivo foi realizado para preencher o cargo de

Posteriormente, durante a execução do projeto foi detectado a necessidade de um
suporte para a demanda burocrática do processo administrativo para o projeto que
até então não estava no escopo da proposta inicial encaminhada a ANEEL.
Portanto, um outro processo seletivo foi realizado para preencher o cargo de

Assistente Administrativo sendo selecionada Alana Monteiro com admissão em 03
de março de 2013, respeitando-se o regimento interno da Fundação para seleção.

Portanto, a equipe que finalizou o projeto e que consta no xml final é a
seguinte:

\begin{itemize}
\item Ramon Romankevicius Costa (CO);
\item Gabriel Alcantara Costa Silva (PE);
\item Eduardo Elael de Melo Soares (PE);
\item Renan Sales de Freitas (PE);
\item Alana Monteiro Lima (AA);
\item Julia Ramos Campana (PE);
\item Andre Abido Figueiró (PE);
\item Alessandro Peixoto Jacoud (PE);
\end{itemize}

Importante deixar registrado que no âmbito dos projetos de Pesquisa e
Desenvolvimento, a Fundação COPPETEC pode fornecer um suporte ao projeto
para seleção de pessoal, entretanto, o critério para a seleção e a escolha da
equipe, segundo a política da Fundação, fica a cargo do coordenador do projeto.
As restrições são analisadas pela Comissão de Ética, que recomenda uma
composição da equipe, responsável pelo desenvolvimento do projeto
predominantemente envolvida com a Universidade, ou seja, é composta por
professores, alunos e técnicos da Universidade.

Segue alguns dos critérios que são adotados pelo regimento interno da Fundação
para a seleção:

Seleção de aluno de Iniciação Cientifica
\begin{itemize}
  \item Estar regulamente matriculado;
  \item Ter boa avalição acadêmica;
  \item Ter experiência (caso seja necessário para área de atuação) e;
  \item Ter noções de língua inglesa.
 \end{itemize}
  
Seleção de aluno Mestrado
\begin{itemize}
  \item Estar regulamente matriculado em um dos programas de pós-gradução;
  \item Ter boa avaliação acadêmica
  \item Ter possibilidade de alinhar seu tema na linha de pesquisa do projeto
contratante;
  \item Ter experiência com projetos de Pesquisa e Desenvolvimento e;
  \item Ter inglês fluente.
\end{itemize}

Seleção de Pesquisador 
\begin{itemize}
  \item Ter ensino superior reconhecido pelo MEC;
  \item Estar regularmente matriculado em um dos programas de pós-graduação;
  \item Ter experiência profissional na área do projeto;
  \item Dominar as ferramentas de trabalho (dependendo do escopo do projeto, as
habilidades dos candidatos podem ser avaliadas para garantir que possuam o
conhecimento necessário para a execução do projeto) e;
  \item Inglês fluente. 
\end{itemize} 

Seleção de Assistente Administrativo
\begin{itemize}
  \item Ter diploma de técnico em administração;
  \item Estar cursando Graduação em Administração;
  \item Ter experiência na área de Administração
  \item Dominar as ferramentas pertinentes ao proposto no escopo do projeto (as
habilidades dos candidatos podem ser avaliadas para garantir que possuam o
conhecimento necessário para a execução do projeto) e;
  \item Ter noções de inglês.
\end{itemize}

Aproveitamos todo esse contexto para mostrar que assim como todo o projeto tem
suas dificuldades em recrutar e até mesmo manter um recurso humano, no projeto
de P\&D não é muito diferente, enfrentamos todas as dificuldades principalmente
pela dificuldade em manter um pesquisador frente as várias ofertas de trabalhos
que os pesquisadores recebem de empresas''.

“Tendo em vista a necessidade de testes finais in loco com os equipamentos e
toda a eletrônica do projeto, aproveitou-se a oportunidade de parada de máquinas
na UHE JIRAU, para constatar a funcionalidade do robô, a fim de apresentar os
testes práticos e consolidação dos dados obtidos no laboratório de pesquisa até
então.
Porém, o período das paradas das Unidades Geradoras na UHE Jirau o qual
seriam aproveitados para realização dos testes, conforme supracitado, para a
certificação da funcionalidade do robô coincidiu com o feriado local da cidade do
Rio de Janeiro sede da COPPETEC/UFRJ, ou seja, as atividades de testes
compreenderam o período de 19/01/15 a 23/01/2015 sendo que o dia 20/01/15
terça-feira foi feriado de dia de São Sebastião Padroeiro da cidade do Rio de
Janeiro, no qual a segunda-feira dia 19/01/2015 foi ponto facultativo no período
classificado como feriado e recesso na cidade do Rio de Janeiro gerando assim
um pequeno excedente das horas, ocasionado justamente pelo feriado na cidade
do Rio de Janeiro. Importante ressaltar que até então não tínhamos autorização
para parada de máquina em outro período para realização do trabalho dos
pesquisadores.
Importante ressaltar que a UHE JIRAU (localizado na cidade de Porto Velho,
Estado de Rondônia) possui um índice de disponibilidade contratual de 99,5%
determinado no edital do leilão A-5 de concessão do empreendimento. Portanto,
devido a importância da contribuição da geração da UHE no Sistema Interligado
Nacional (SIN), e diante do supracitado não se podia parar a Unidade Geradora
sem autorização da Operador Nacional do Sistema e consequentemente da
ANEEL, pois qualquer indisponibilidade poderia acarretar em prejuízos no SIN.
Diante dos fatos expostos, não se tinha a previsão exata do próximo período de
parada de máquinas na UHE para que a equipe pudesse aproveitar a oportunidade
para realizar os testes do robô. E, a possibilidade de postergar a viagem poderia
trazer sérios prejuízos no desenvolvimento dos resultados da pesquisa podendo
até comprometê-la.
Vale ressaltar ainda que a Usina Hidrelétrica Jirau foi projetada para manter alta
performance de suas turbinas, nas diferentes vazões do Rio Madeira, ou seja,
opera normalmente tanto em períodos de seca, quanto em períodos de cheias
extraordinárias.

Mesmo durante a motorização da Usina, a UHE Jirau tem mantido seus índices
de disponibilidade contratual muito próximos do percentual de 99,5%
determinados no edital do leilão de concessão do empreendimento. O projeto de
construção civil, a tecnologia das turbinas adotadas pela ESBR e os estudos
aprofundados do comportamento do Rio Madeira, possibilitaram colocar, apenas
no ano de 2014, 22 Unidades Geradoras em operação. Um feito inédito para o
Brasil, que acrescentou 1.650 MW de potência ao SIN no ano passado.
Em 2016, ano de conclusão da Hidrelétrica, a Usina Jirau vai operar 50 UGs. Ao
todo, sua capacidade instalada de 3.750 MW, será suficiente para abastecer mais
de 10 milhões de residências por todo o país.
Aproveita-se a oportunidade para salientar que não houve nenhum tipo de
excedentes nos próximos meses de execução do projeto até seu término”.

Segue na planilha abaixo os demonstrativos dos custos do projeto, ou seja, os
valores previstos incialmente no xml inicial e os valores realizados no
projeto, bem como as devidas justificativas para as diferenças entre o previsto versus realizado.


\includepdf{PDFs/Tabela_Gastos.pdf}
 
\section{Equipe de desenvolvimento do projeto}

\section{Financeiro}
\begin{center}
  \begin{tabular}{ | c | c | c | c | p{4.9cm} | }
    \hline
        \textbf{Rúbrica} & \textbf{Valor Previsto (R\$)} & \textbf{Valor Realizado (R\$)} & \textbf{Devio (\%)} & \textbf{Justificativa para os desvios positivos} \\ \hline
    RH & 1.207.409,64 & 975.288,71 & 95\%  &  \\ \hline
    MC & 2500,00      & 11.314,90  & 453\% &  Custo foi subestimado na estimativa  inicial dos componentes para construção do protótipo \\ \hline
    MP & 1.972.875,00 & 987.282,99 & 50\%  &  \\ \hline
    ST & 700.000,00   & 433.904,15 & 62\%  &  \\ \hline
    VD & 158.160,00   & 95.871,00  & 61\%  &  \\ \hline
    OU & 503.273,14   & 376.338,75 & 75\%  &  \\
    \hline
  \end{tabular}
\end{center}

\standardchapterstyle
\bibliographystyle{ieeetr}
\bibliography{resumo}

\fancychapterstyle  

% ---------------------------------------------------------------------------
% ---------------------------------------------------------------------------



 
 
% ---------------------------------------------------------------------------
% ---------------------------------------------------------------------------
%\chapter{Registros Fotográficos}
%\section{Relatório de Viagem Jirau - Novembro 2013}
Data: 13 a 16 de novembro de 2013.
Equipe: Julia Campana, Ramon Costa e Alessandro Jacoud.


Viagem de reconhecimento de campo com alguns integrantes do projeto para
familiarização do ambiente e primeiras impressões técnicas a fim dar inicio aos
trabalhos de projeção e montagem do Robô Rosa. Houve um intercambio importante
de informções onde foram apresentados estudos de pesquisa realizados pela ESBR
que tem relevància com o Projeto Rosa.



\begin{figure}[h!]
\centering
  \includegraphics[width=1\linewidth]{Fotos/Novembro2013/1.jpg}
  \caption{nov20131}
  \label{nov20131}
\end{figure}

\begin{figure}[h!]
  \centering
  \includegraphics[width=1\linewidth]{Fotos/Novembro2013/2.jpg}
  \caption{1b}
  \label{nov20132}
\end{figure}

\begin{figure}[h!]
  \centering
  \includegraphics[width=1\linewidth]{Fotos/Novembro2013/3.jpg}
  \caption{nov20133}
  \label{nov20133}
\end{figure}

\begin{figure}[h!]
  \centering
  \includegraphics[width=1\linewidth]{Fotos/Novembro2013/4.jpg}
  \caption{1b}
  \label{nov20134}
\end{figure}
%\section{Solenidade}

\begin{figure}[h!]
\centering
  \includegraphics[width=1\linewidth]{Fotos/Solenidade/1.jpg}
  \caption{nov20131}
  \label{nov20131}
\end{figure}

\begin{figure}[h!]
  \centering
  \includegraphics[width=1\linewidth]{Fotos/Solenidade/2.jpg}
  \caption{1b}
  \label{nov20132}
\end{figure}

\begin{figure}[h!]
  \centering
  \includegraphics[width=1\linewidth]{Fotos/Solenidade/3.jpg}
  \caption{nov20133}
  \label{nov20133}
\end{figure}

\begin{figure}[h!]
  \centering
  \includegraphics[width=1\linewidth]{Fotos/Solenidade/4.jpg}
  \caption{1b}
  \label{nov20134}
\end{figure}

\begin{figure}[h!]
  \centering
  \includegraphics[width=1\linewidth]{Fotos/Solenidade/5.jpg}
  \caption{1b}
  \label{nov20134}
\end{figure}

\begin{figure}[h!]
  \centering
  \includegraphics[width=1\linewidth]{Fotos/Solenidade/6.jpg}
  \caption{1b}
  \label{nov20135}
\end{figure}

\begin{figure}[h!]
  \centering
  \includegraphics[width=1\linewidth]{Fotos/Solenidade/7.jpg}
  \caption{1b}
  \label{nov20136}
\end{figure}
%\section{Teste no LabOceano}

\begin{figure}[h!]
\centering
  \includegraphics[width=1\linewidth]{Fotos/TesteLabOceanico/1.jpg}
  \caption{nov20131}
  \label{nov20131}
\end{figure}

\begin{figure}[h!]
  \centering
  \includegraphics[width=1\linewidth]{Fotos/TesteLabOceanico/2.jpg}
  \caption{1b}
  \label{nov20132}
\end{figure}
%\section{Viagem Jirau - Junho 2014}

\begin{figure}[h!]
\centering
  \includegraphics[width=1\linewidth]{Fotos/JirauJunho2014/1.jpg}
  \caption{nov20131}
  \label{nov20131}
\end{figure}

\begin{figure}[h!]
  \centering
  \includegraphics[width=1\linewidth]{Fotos/JirauJunho2014/2.jpg}
  \caption{1b}
  \label{nov20132}
\end{figure}

\begin{figure}[h!]
  \centering
  \includegraphics[width=1\linewidth]{Fotos/JirauJunho2014/3.jpg}
  \caption{nov20133}
  \label{nov20133}
\end{figure}

\begin{figure}[h!]
  \centering
  \includegraphics[width=1\linewidth]{Fotos/JirauJunho2014/4.jpg}
  \caption{1b}
  \label{nov20134}
\end{figure}

\begin{figure}[h!]
  \centering
  \includegraphics[width=1\linewidth]{Fotos/JirauJunho2014/5.jpg}
  \caption{1b}
  \label{nov20134}
\end{figure}

\begin{figure}[h!]
  \centering
  \includegraphics[width=1\linewidth]{Fotos/JirauJunho2014/6.jpg}
  \caption{1b}
  \label{nov20135}
\end{figure}

\begin{figure}[h!]
  \centering
  \includegraphics[width=1\linewidth]{Fotos/JirauJunho2014/7.jpg}
  \caption{1b}
  \label{nov20136}
\end{figure}

\begin{figure}[h!]
  \centering
  \includegraphics[width=1\linewidth]{Fotos/JirauJunho2014/8.jpg}
  \caption{1b}
  \label{nov20136}
\end{figure}

\begin{figure}[h!]
  \centering
  \includegraphics[width=1\linewidth]{Fotos/JirauJunho2014/9.jpg}
  \caption{1b}
  \label{nov20136}
\end{figure}

\begin{figure}[h!]
  \centering
  \includegraphics[width=1\linewidth]{Fotos/JirauJunho2014/10.jpg}
  \caption{1b}
  \label{nov20136}
\end{figure}

\begin{figure}[h!]
  \centering
  \includegraphics[width=1\linewidth]{Fotos/JirauJunho2014/11.jpg}
  \caption{1b}
  \label{nov20136}
\end{figure}
%\section{Relatório de Viagem Jirau - Novembro 2014}
Data: 03 a 07 de Novembro de 2014.

Equipe: Eduardo Elael, Gabriel Alcântara, Julia Campana, Renan Salles e Rodrigo Carneiro.


Viagem de equipe do Projeto ROSA a UHE Jirau, realizada entre
os dia 3 e 7 de Novembro de 2014. O objetivo foi realizar testes de campo dos
componentes do Robô ROSA assim como os testes de usabilidade do aplicativo ROSA
- sistema que monitora as operações de inserção e remoção de stoplogs.

Como resultado dos testes de componentes, conclui-se que a posição usada na
garra ajustada de acordo com dados da viagem anterior funcionou com
perfeitamente, assim como  os sensores indutivos que também se comportaram da
forma esperada nos fornecendo dados necessários.  Também foram realizados
testes do Sonar modelo Seaking funcionou de acordo e nos forneceu dados a serem
processados e futramente alinhados com o sistema do robô.

No quesito usabilidade tivemos testes importantes com o operador da viga, que
utilizou pela primeira vez o sistema durante a operação, foram aplicados
testes de usabilidade, heurísticos e também uma entrevista para feedback
informal com realação a familiaridade e linguagem visual.

As figuras~\ref{nov20141}, ~\ref{nov20142}, ~\ref{nov20143}, ~\ref{nov20144},
~\ref{nov20145} foram obtidas durante a viagem de Novembro, 2014.


\begin{figure}[h!]
\centering
  \includegraphics[width=1\linewidth]{Fotos/Novembro2014/1.JPG}
  \caption{Montagem de sensores na garra pesacdora.}
  \label{nov20141}
\end{figure}

\begin{figure}[h!]
  \centering
  \includegraphics[width=1\linewidth]{Fotos/Novembro2014/3.JPG}
  \caption{Eletrônica embarcada acoplada a viga pescadora.}
  \label{nov20142}
\end{figure}

\begin{figure}[h!]
  \centering
  \includegraphics[width=1\linewidth]{Fotos/Novembro2014/4.JPG}
  \caption{Engenheiro fazendo ajustes de componentes.}
  \label{nov20143}
\end{figure}

\begin{figure}[h!]
  \centering
  \includegraphics[width=1\linewidth]{Fotos/Novembro2014/6.JPG}
  \caption{Sensor indutivo acoplado a garra pescadora.}
  \label{nov20144}
\end{figure}

\begin{figure}[h!]
  \centering
  \includegraphics[width=1\linewidth]{Fotos/Novembro2014/10.JPG}
  \caption{Operador em ação operando a viga pescadora.}
  \label{nov20145}
\end{figure}
%\section{Relatório de Viagem Jirau - Janeiro 2015}
Data: 19 a 23 de Janeiro de 2015.
Equipe: Eduardo Elael, Gabriel Alcântara, Julia Campana, Renan Salles e Rodrigo Carneiro e Ramon Costa.

Viagem de equipe do Projeto ROSA a UHE Jirau, realizada entre
os dia 19 e 22 de Janeiro de 2015. O objetivo foi realizar testes de campo dos
componentes do Robô ROSA assim cmo novos testes de usabilidade do aplicativo
operacional ROSA.

Como resultado dos testes de componentes (Sonar Seaking, Pan Tilt, eletrônica
embarcada, sensores indutivos, inclinômetros e sensor de pressão) foi possível
coletar dados de funcionamento condizentes como esperado.

No quesito de usabilidade, segunda versao do alpcativo, modificada de acordo com
os resultados dos testes prévios se mostraou eficaz, tendo agora a adição de
sons que reafirmar o status de sucesso ou falha na operação o que trouxe grande
ganho para o usuário. Na parte heurística foi possível quantificar tais
melhoras e adicionar numeros conclusivos a nossa pesquisa.



\begin{figure}[h!]
\centering
  \includegraphics[width=1\linewidth]{Fotos/Janeiro2015/5.jpg}
  \caption{Engenheiros instalando equipamentos.}
  \label{nov20131}
\end{figure}

\begin{figure}[h!]
  \centering
  \includegraphics[width=1\linewidth]{Fotos/Janeiro2015/9.JPG}
  \caption{Medição para montagem.}
  \label{nov20132}
\end{figure}

\begin{figure}[h!]
  \centering
  \includegraphics[width=1\linewidth]{Fotos/Janeiro2015/13.JPG}
  \caption{Viga pescadora pronta para ser imersa com equipamento acoplado.}
  \label{nov20133}
\end{figure}

\begin{figure}[h!]
  \centering
  \includegraphics[width=1\linewidth]{Fotos/Janeiro2015/20.JPG}
  \caption{Equipe do projeto rosa na cabine do operador realizando testes de
  usabilidade do aplicativo.}
  \label{nov20134}
\end{figure}

\begin{figure}[h!]
  \centering
  \includegraphics[width=1\linewidth]{Fotos/Janeiro2015/23.JPG}
  \caption{Equipes da UFRJ e da ESBR reunidas.}
  \label{nov20134}
\end{figure}

\begin{figure}[h!]
  \centering
  \includegraphics[width=1\linewidth]{Fotos/Janeiro2015/24.JPG}
  \caption{Montagem para inspeção com Sonar.}
  \label{nov20135}
\end{figure}

\begin{figure}[h!]
  \centering
  \includegraphics[width=1\linewidth]{Fotos/Janeiro2015/28.JPG}
  \caption{Comportas da margem esquerda da usina.}
  \label{nov20136}
\end{figure}


% ---------------------------------------------------------------------------
% ---------------------------------------------------------------------------

\chapter{Projeto Básico}
\includepdf[pages=2-,
addtotoc={9,section,1,Nomenclatura,nomrelgel[,14,section,1,Introdução,introrelgel[,15,section,1,Descriçãodo
problema,desrelgel[,15,subsection,2,Viagem de
Reconhecimento,viarelgel[,24,subsection,2,Operação padrão de
inserção,operelgel[,25,subsection,2,Operação excepcional de inserção 1 -
Travamento durante inserção,ope1relgel[,26,subsection,2,Operação excepcional de
inserção 2 - Falha do desencaixe da garra
pescadora,ope2relgel[,27,subsection,2,Operação excepcional de inserção 3 - Não
vedamento devido ao acúmulo de detritos na base do
trilho,ope3relgel[,27,subsection,2,Operação padrão -
Remoção,operemrelgel[,28,subsection,2,Operação excepcional de remoção 1 - falha
no encaixe, operem1relgel[,29,subsection,2,Operação excepcional de remoção 2 -
Travamento durante remoção, operem2relgel[,29,subsection,2,Operação excepcional
de remoção 3 - Acúmulo de sedimentos no fundo,
operem3relgel[,31,section,1,Metodologia,metrelgel[,32,section,1,Pesquisa
Bibliogáica,
pesrelgel[,32,subsection,2,Stoplogs,stoplogsrelgel[,43,section,1,Projeto
Conceitual,projconrelgel[,43,subsection,2,Operação excepcional de inserção 1 -
Travamento durante inserção,ope12relgel[,44,subsection,2,Operação excepcional de
inserção 2 - Falha do desencaixe da garra
pescadora,ope22relgel[,45,subsection,2,Operação excepcional de inserção 3 - Não
vedamento devido ao acúmulo de detritos na base do trilho,
ope13relgel[,46,subsection,2,Operação excepcional de remoção 1 - Falha no
encaixe,ope12erelgel[,46,subsection,2,Operação excepcional de remoção 2 -
Travamento durante remoção, operem22relgel[,47,subsection,2,Operação excepcional
de remoção 3 - Acúmulo de sedimentos no
fundo,operem32relgel[,47,subsection,2,Conclusão do Conceito
Básico,concb2relgel[,49,section,1,Pesquisa
tecnológica,pestecrelgel[,50,subsection,2,Sensores de Contato,
sencontrelgel[,50,subsubsection,3,Sensor de
força,senforelgel[,52,subsubsection,3,Sensor indutivo de
proximidade,senindprorelgel[,54,subsubsection,3,Sensor capacitivo de
proximidade,sencaprelgel[,56,subsubsection,3,Conclusão de análise
técnica,contecrelgel[,58,subsection,2,Posição
Angular,posangrelgel[,58,subsubsection,3,Encoder,encrelgel[,59,subsubsection,3,Conclusão
de análise técnica,contec2relgel[,61,subsection,2,Mapeamento 3D,
maprelgel[,61,subsubsection,3,Sonar,sonrelgel[,66,subsubsection,3,Unidade Pan e
Tilt,pturelgel[,67,subsection,2,Reconstrução de superfície
3D,recsuprelgel[,72,subsection,2,Sistema de Gerenciamento de
Umbilical,sisgenrelgel[,74,section,1,Sistema
proposto,sisprorelgel[,74,subsection,2,Operação padrão (inspeção e
remoção),opepadinsremrelgel[,74,subsubsection,3,Sensores,sensoresrelgel[,75,subsubsection,3,Eletrônica
embarcada,eletembrelgel[,78,subsubsection,3,Umbilical,umbrelgel[,78,subsubsection,3,Eletrônica
de superfície - base, elesupbaserelgel[,78,subsubsection,3,Eletrônica de
superfície - remota,
elesupremrelgel[,78,subsubsection,3,Carretel,carretelrelgel[,79,subsection,2,Operação
Excepcional 1 - Inspeção,opeexcprelgel[,80,subsection,2,Operação Excepcional 2
- Remoção de sedimentos sobre olhal,olhalrelgel[,81,section,1,Fluxograma da
solução,fluxrelgel]]]]]]]]]]]]]]]]]]]]]]]]]]]]]]]]]]]]]]]]]]]]]]]}]{PDFs/RelGeral.pdf}
% ---------------------------------------------------------------------------
% ---------------------------------------------------------------------------

\chapter{Acompanhamentos e testes}
A abordagem metodológica do projeto teve início na elaboração da solução do problema. Após o
detalhamento do conceito base, foram planejadas as fases de desenvolvimento e execução que
abrangeram um total de 4 testes ao longo do projeto. Estes testes foram executados em campo na
UHE Jirau, por toda a equipe de pesquisadores do projeto durante um período de 5
dias a cada viagem realizada.

\emph{Teste 1:}
Realizado entre os 02/06/2015 a 06/06/2015 o primeiro teste de campo teve como objetivo verifi-
car a estrutura necessária para cabeamento e instalação do robô com seus dispositivos na viga
pescadora. Foi realizada a montagem, com treinamento do pessoal da UHE Jirau, e
a validação da funcionalidade do protótipo.

\emph{Teste 2:}
Realizado entre os 03/11/2014 a 07/11/2014 o segundo teste de campo teve como objetivo testar
os dispositivos do robô individualmente. Foram realizados testes com o Sonar
modelo SeaKing da empresa Tritech na soleira disponível, testes de sensores
indutivos nas garras da viga pescadora, e testes heurísticos de usabilidade
com a adição de sons no aplicativo dentro da cabine do operador.

\emph{Teste 3:}
Realizado entre os 19/01/2015 a 23/01/2015 o terceiro teste de campo teve como objetivo testar
o funcionamento do robô em sua totalidade. Foram realizados testes do sistema com todos seus
dispositivos ativos, bem como o teste heurístico e acompanhamento de atividades de dentro da
cabine do operador.

\emph{Teste 4:}
Realizado entre os 23/02/2015 a 27/02/2015 o último teste de campo teve como objetivo testar
todos os sistemas do robô ROSA, treinamento de usabilidade e amostragem final do projeto.

% ---------------------------------------------------------------------------
% ---------------------------------------------------------------------------

\chapter{Melhorias de processos}\label{chap::melpro}
O objetivo do robô ROSA é de otimizar o processo de inserção e remoção de
stoplogs, operações frequentes para atividades de manutenção na usina. A
importância dessa otimização se dá à medida que o tempo de parada de máquinas
pode diminuir consideravelmente com o monitoramento do processo, assim como
acidentes de operação. Através deste monitoramento é possível identificar
problemas mecânicos como o emperramento nos trilhos e falhas de pesca de
stoplogs, assim como problemas oriundos do contexto ambiental da usina, como o
grande volume de detritos acumulados nas soleiras que impedem o perfeito
assentamento dos stoplogs.

Ao abordar o problema foi entendido que o processo se dá de forma puramente
mecânica, onde o operador obtém feedback através de barulhos do motor e mudança
na tensão dos cabos que seguram os stoplogs. Desta forma, sensores indutivos,
inclinômetro, profundímetro foram estruturados na viga pescadora para monitorar
seu funcionamento e dar ao operador um contexto atualizado do seu ambiente de
operação. Assim o aplicativo ROSA, visualizado em um tablet dentro da cabine do
operador, permite um feedback visual e sonoro das operações realizadas.

Desta forma o aplicativo facilita o processo de remoção e inserção de stoplog a
medida que aponta erros comuns na operação que antes não podiam ser determinados com
precisão. Ao antever possíveis problemas, é possível mitigar alguns processos e
dar prosseguimento a manutenção, aumentando a disponibilidade das máquinas e,
consequentemente, fornecendo energia para o sistema elétrico brasileiro


% ---------------------------------------------------------------------------
% ---------------------------------------------------------------------------

\chapter{Melhorias introduzidas nos equipamentos}
O sistema robótico desenvolvido requer uma pequena modificação permanente na
viga pescadora: o acoplamento de sensores indutivos nas garras em locais
estratégicos para a percepção de proximidade das garras com os olhais do
stoplog, garantindo o monitoramento da ocorrên- cia de pega ou liberação dos
stoplogs. Essa pequena alteração não prejudica em nada a operação e, em
conjunto com o sistema, auxilia no processo, como foi elucidado no
capítulo~\ref{chap::melpro}. Vale ressaltar que esta é uma melhoria
introduzida a um equipamento frequentemente utilizado na usina, pois os
sensores indutivos podem ser futuramente utilizados com outras eletrônicas e/ou
até com sistemas mais simples que visam utilizar parcialmente as
funcionalidades do sistema robótico ROSA desenvolvido. Tornando a viga
pescadora em uma viga inteligente.


% ---------------------------------------------------------------------------
% ---------------------------------------------------------------------------

\chapter{Relatório da elétrica e eletrônica}
\includepdf[pages=3-, addtotoc={5,section,1,Propostas de soluções para a
arquitetura da eletrônica do Projeto ROSA versão
1,versao1ele[,6,section,1,Proposta 1 - Placa com Microcontrolador e Gateway
Ethernet, prop1ele[,6,subsection,2,Arquitetura da Eletrônica Proposta 1 - versão
1,prop1v1ele[,10,subsection,2,Arquitetura da Eletrônica Proposta 1 - versão
2,prop1v2ele[,16,subsection,2,Arquitetura da Eletrônica Proposta 1 - versão 3,
prop1v3ele[,19,subsection,2,Arquitetura de Software Proposta 1 - versão
1,props1v1ele[,23,section,1,Proposta 2 - PC Embarcado e base com
Roteador,prop2ele[,23,subsection,2,Arquitetura da Eletrônica Proposta
2,prop2eleele[,24,subsection,2,Arquitetura de Software proposta
2,prop2sofele[,27,section,1,Proposta 3 - PC embarcado e PC na
base,prop3ele[,27,subsection,2,Arquitetura da Eletrônica Proposta
3,prop3elele[,27,subsection,2,Arquitetura de Software Proposta 3,
prop3softele[,33,section,1,Datasheets,dataele]]]]]]]]]]]}]{PDFs/RelEle.pdf}
% ---------------------------------------------------------------------------
% ---------------------------------------------------------------------------
\chapter{Relatório de usabilidade}
\includepdf[pages=-]{PDFs/RelatorioUsabilidade.pdf}

% ---------------------------------------------------------------------------
% ---------------------------------------------------------------------------
\chapter{Relatórios de viagens} 

\section{Relatório de Viagem Jirau - Novembro 2013}
Data: 13 a 16 de novembro de 2013.
Equipe: Julia Campana, Ramon Costa e Alessandro Jacoud.


Viagem de reconhecimento de campo com alguns integrantes do projeto para
familiarização do ambiente e primeiras impressões técnicas a fim dar inicio aos
trabalhos de projeção e montagem do Robô Rosa. Houve um intercambio importante
de informções onde foram apresentados estudos de pesquisa realizados pela ESBR
que tem relevància com o Projeto Rosa.



\begin{figure}[h!]
\centering
  \includegraphics[width=1\linewidth]{Fotos/Novembro2013/1.jpg}
  \caption{nov20131}
  \label{nov20131}
\end{figure}

\begin{figure}[h!]
  \centering
  \includegraphics[width=1\linewidth]{Fotos/Novembro2013/2.jpg}
  \caption{1b}
  \label{nov20132}
\end{figure}

\begin{figure}[h!]
  \centering
  \includegraphics[width=1\linewidth]{Fotos/Novembro2013/3.jpg}
  \caption{nov20133}
  \label{nov20133}
\end{figure}

\begin{figure}[h!]
  \centering
  \includegraphics[width=1\linewidth]{Fotos/Novembro2013/4.jpg}
  \caption{1b}
  \label{nov20134}
\end{figure}

\includepdf[pages=-,addtotoc={1,section,1,Viagem de Junho 2014,junhoviagem}]{PDFs/01ViagemJunho2014.pdf}
\section{Viagem Jirau - Junho 2014}

\begin{figure}[h!]
\centering
  \includegraphics[width=1\linewidth]{Fotos/JirauJunho2014/1.jpg}
  \caption{nov20131}
  \label{nov20131}
\end{figure}

\begin{figure}[h!]
  \centering
  \includegraphics[width=1\linewidth]{Fotos/JirauJunho2014/2.jpg}
  \caption{1b}
  \label{nov20132}
\end{figure}

\begin{figure}[h!]
  \centering
  \includegraphics[width=1\linewidth]{Fotos/JirauJunho2014/3.jpg}
  \caption{nov20133}
  \label{nov20133}
\end{figure}

\begin{figure}[h!]
  \centering
  \includegraphics[width=1\linewidth]{Fotos/JirauJunho2014/4.jpg}
  \caption{1b}
  \label{nov20134}
\end{figure}

\begin{figure}[h!]
  \centering
  \includegraphics[width=1\linewidth]{Fotos/JirauJunho2014/5.jpg}
  \caption{1b}
  \label{nov20134}
\end{figure}

\begin{figure}[h!]
  \centering
  \includegraphics[width=1\linewidth]{Fotos/JirauJunho2014/6.jpg}
  \caption{1b}
  \label{nov20135}
\end{figure}

\begin{figure}[h!]
  \centering
  \includegraphics[width=1\linewidth]{Fotos/JirauJunho2014/7.jpg}
  \caption{1b}
  \label{nov20136}
\end{figure}

\begin{figure}[h!]
  \centering
  \includegraphics[width=1\linewidth]{Fotos/JirauJunho2014/8.jpg}
  \caption{1b}
  \label{nov20136}
\end{figure}

\begin{figure}[h!]
  \centering
  \includegraphics[width=1\linewidth]{Fotos/JirauJunho2014/9.jpg}
  \caption{1b}
  \label{nov20136}
\end{figure}

\begin{figure}[h!]
  \centering
  \includegraphics[width=1\linewidth]{Fotos/JirauJunho2014/10.jpg}
  \caption{1b}
  \label{nov20136}
\end{figure}

\begin{figure}[h!]
  \centering
  \includegraphics[width=1\linewidth]{Fotos/JirauJunho2014/11.jpg}
  \caption{1b}
  \label{nov20136}
\end{figure}
%\includepdf[pages=3-,addtotoc={3,section,1,Relatório Técnico
%Viagem de Junho 2014,tecjunhoviagem}]{PDFs/01ViagemTecnicoJunho2014.pdf}

\includepdf[pages=-,addtotoc={1,section,1, Viagem de Novembro
2014,novviagem}]{PDFs/02ViagemNovembro2014.pdf}
\section{Relatório de Viagem Jirau - Novembro 2014}
Data: 03 a 07 de Novembro de 2014.

Equipe: Eduardo Elael, Gabriel Alcântara, Julia Campana, Renan Salles e Rodrigo Carneiro.


Viagem de equipe do Projeto ROSA a UHE Jirau, realizada entre
os dia 3 e 7 de Novembro de 2014. O objetivo foi realizar testes de campo dos
componentes do Robô ROSA assim como os testes de usabilidade do aplicativo ROSA
- sistema que monitora as operações de inserção e remoção de stoplogs.

Como resultado dos testes de componentes, conclui-se que a posição usada na
garra ajustada de acordo com dados da viagem anterior funcionou com
perfeitamente, assim como  os sensores indutivos que também se comportaram da
forma esperada nos fornecendo dados necessários.  Também foram realizados
testes do Sonar modelo Seaking funcionou de acordo e nos forneceu dados a serem
processados e futramente alinhados com o sistema do robô.

No quesito usabilidade tivemos testes importantes com o operador da viga, que
utilizou pela primeira vez o sistema durante a operação, foram aplicados
testes de usabilidade, heurísticos e também uma entrevista para feedback
informal com realação a familiaridade e linguagem visual.

As figuras~\ref{nov20141}, ~\ref{nov20142}, ~\ref{nov20143}, ~\ref{nov20144},
~\ref{nov20145} foram obtidas durante a viagem de Novembro, 2014.


\begin{figure}[h!]
\centering
  \includegraphics[width=1\linewidth]{Fotos/Novembro2014/1.JPG}
  \caption{Montagem de sensores na garra pesacdora.}
  \label{nov20141}
\end{figure}

\begin{figure}[h!]
  \centering
  \includegraphics[width=1\linewidth]{Fotos/Novembro2014/3.JPG}
  \caption{Eletrônica embarcada acoplada a viga pescadora.}
  \label{nov20142}
\end{figure}

\begin{figure}[h!]
  \centering
  \includegraphics[width=1\linewidth]{Fotos/Novembro2014/4.JPG}
  \caption{Engenheiro fazendo ajustes de componentes.}
  \label{nov20143}
\end{figure}

\begin{figure}[h!]
  \centering
  \includegraphics[width=1\linewidth]{Fotos/Novembro2014/6.JPG}
  \caption{Sensor indutivo acoplado a garra pescadora.}
  \label{nov20144}
\end{figure}

\begin{figure}[h!]
  \centering
  \includegraphics[width=1\linewidth]{Fotos/Novembro2014/10.JPG}
  \caption{Operador em ação operando a viga pescadora.}
  \label{nov20145}
\end{figure}
%\includepdf[pages=-,addtotoc={1,section,1,Relatório Técnico
%Viagem de Novembro 2014,tecnovviagem}]{PDFs/02ViagemTecnicoNovembro2014.pdf} 

\includepdf[pages=1,addtotoc={1,section,1,Viagem de Janeiro 2015,janviagem}]{PDFs/03ViagemJaneiro2015.pdf}
\section{Relatório de Viagem Jirau - Janeiro 2015}
Data: 19 a 23 de Janeiro de 2015.
Equipe: Eduardo Elael, Gabriel Alcântara, Julia Campana, Renan Salles e Rodrigo Carneiro e Ramon Costa.

Viagem de equipe do Projeto ROSA a UHE Jirau, realizada entre
os dia 19 e 22 de Janeiro de 2015. O objetivo foi realizar testes de campo dos
componentes do Robô ROSA assim cmo novos testes de usabilidade do aplicativo
operacional ROSA.

Como resultado dos testes de componentes (Sonar Seaking, Pan Tilt, eletrônica
embarcada, sensores indutivos, inclinômetros e sensor de pressão) foi possível
coletar dados de funcionamento condizentes como esperado.

No quesito de usabilidade, segunda versao do alpcativo, modificada de acordo com
os resultados dos testes prévios se mostraou eficaz, tendo agora a adição de
sons que reafirmar o status de sucesso ou falha na operação o que trouxe grande
ganho para o usuário. Na parte heurística foi possível quantificar tais
melhoras e adicionar numeros conclusivos a nossa pesquisa.



\begin{figure}[h!]
\centering
  \includegraphics[width=1\linewidth]{Fotos/Janeiro2015/5.jpg}
  \caption{Engenheiros instalando equipamentos.}
  \label{nov20131}
\end{figure}

\begin{figure}[h!]
  \centering
  \includegraphics[width=1\linewidth]{Fotos/Janeiro2015/9.JPG}
  \caption{Medição para montagem.}
  \label{nov20132}
\end{figure}

\begin{figure}[h!]
  \centering
  \includegraphics[width=1\linewidth]{Fotos/Janeiro2015/13.JPG}
  \caption{Viga pescadora pronta para ser imersa com equipamento acoplado.}
  \label{nov20133}
\end{figure}

\begin{figure}[h!]
  \centering
  \includegraphics[width=1\linewidth]{Fotos/Janeiro2015/20.JPG}
  \caption{Equipe do projeto rosa na cabine do operador realizando testes de
  usabilidade do aplicativo.}
  \label{nov20134}
\end{figure}

\begin{figure}[h!]
  \centering
  \includegraphics[width=1\linewidth]{Fotos/Janeiro2015/23.JPG}
  \caption{Equipes da UFRJ e da ESBR reunidas.}
  \label{nov20134}
\end{figure}

\begin{figure}[h!]
  \centering
  \includegraphics[width=1\linewidth]{Fotos/Janeiro2015/24.JPG}
  \caption{Montagem para inspeção com Sonar.}
  \label{nov20135}
\end{figure}

\begin{figure}[h!]
  \centering
  \includegraphics[width=1\linewidth]{Fotos/Janeiro2015/28.JPG}
  \caption{Comportas da margem esquerda da usina.}
  \label{nov20136}
\end{figure}
%\includepdf[pages=-,addtotoc={1,section,1,Relatório Técnico
%Viagem de Janeiro 2015,tecjanviagem}]{PDFs/03ViagemTecnicoJaneiro2015.pdf}

\includepdf[pages=-,addtotoc={1,section,1,Relatório
Viagem de Fevereiro 2015,fevviagem}]{PDFs/04ViagemFevereiro2015.pdf}
\section{Relatório de viagem à UHE Jirau
referente a Fevereiro 2015}
Data: 23 a 27 de Fevereiro de 2015.

Equipe: Eduardo Elael, Gabriel Alcântara, Alana Monteiro, Renan Salles e Rodrigo Carneiro e Ramon Costa.


Viagem de equipe do Projeto ROSA a UHE Jirau, realizada entre
os dia 23 e 27 de Fevereiro de 2015. O objetivo foi entregar o protótipo do Robô
ROSA e ministrar um \textit{workshop} relacionado a sua parte técnica e
operacional. A entrega foi um sucesso com testes de içamento de stoplogs fora
d'água e a presença do Diretor de Operação e Manutenção da ESBR, Isaac Teixeira.

Os testes, apesar de semelhante aos executados anteriormente, foi executado com
o último e mais robusto protótipo da eletrônica e umbilical. O dia de testes era
chuvoso, mas o sistema se mostrou robusto ao clima adverso. 

As figuras~\ref{fev20151}, ~\ref{fev20152}, ~\ref{fev20153} foram obtidas durante a viagem
de Fevereiro, 2015. 

\begin{figure}[h!]
\centering
  \includegraphics[width=1\linewidth]{Fotos/Fevereiro2015/fev20151.jpg}
  \caption{Equipe LEAD/COPPETEC realizando a montagem do protótipo final.}
  \label{fev20151}
\end{figure}

\begin{figure}[h!]
  \centering
  \includegraphics[width=1\linewidth]{Fotos/Fevereiro2015/fev20152.jpg}
  \caption{Rodrigo Carneiro, pesquisador LEAD/COPPETEC, dando instruções de
  montagem da eletrônica.}
  \label{fev20152}
\end{figure}

\begin{figure}[h!]
  \centering
  \includegraphics[width=1\linewidth]{Fotos/Fevereiro2015/fev20153.jpg}
  \caption{Equipe LEAD/COPPETEC e ESBR na entrega do protótipo do robô ROSA.}
  \label{fev20153}
\end{figure}
%\includepdf[pages=-,addtotoc={1,section,1,Relatório Técnico
%Viagem de Fevereiro 2015,tecfevviagem}]{PDFs/04ViagemTecnicoFevereiro2015.pdf}


% ---------------------------------------------------------------------------
% ---------------------------------------------------------------------------

\chapter{Capacitação profissional}
A pesquisa e o desenvolvimento (P\&D) tem como propósito fomentar o avanço
tecnológico e novas maneiras de desenvolver um tipos específicos de conhecimento
no país. O desenvolvimento do robô ROSA, no âmbito P\&D é um exemplo de como a
parceria entre agências do governo e uni- versidades federais podem colaborar
para a capacitação tecnológica e o desenvolvimento de novas tecnologias.
Especificamente na área de robótica, o projeto ROSA mostra como otimização e
automação de diversos processos de trabalho pode contribuir na indústria
energética. Em termos acadêmicos, esta linha de projetos já originou três
dissertações de mestrado no Programa de Engenharia Elétrica (PEE), Universidade
Federal do Rio de Janeiro (UFRJ), abordando os seguintes tópicos:

\begin{itemize}
\item André Abido Figueiró, Estudo de Modelagem de Sistemas de Gerenciamento e
Balanceamento de Baterias.
\item Gabriel Silva Alcântara, \textit{Underwater 6 DOF Localization Using
Imaging Sonars}
\item Eduardo Elael, \textit{Underwater Mapping Using Imaging Sonar}
\end{itemize}

Os documentos comprobatórios das inscrições de mestrado seguem abaixo:

\includepdf{PDFs/abelha_mestrado}
\includepdf{PDFs/elael_mestrado}

% ---------------------------------------------------------------------------
% ---------------------------------------------------------------------------

\chapter{Manual de usuário}
\includepdf[pages=-,
addtotoc={3,section,1,Visão geral,geralmanual[,3,subsection,2,O Robô,
robomanual[,5,subsection,2,Sensores,sensoresmanual[,6,section,1,Tipos de
operação,operacaomanual[,7,section,1,Instalação
mecânica,mecanicamanual[,8,subsection,2,Instalação da eletrôncia
embarcada,embarcadamanual[,10,subsection,2,Instalação dos
sensores,instsensoresmanual[,14,subsection,2,Instalação da eletrônica de
superfície,supericiemanual[,15,subsection,2,Instalação cabos e
conectores,conectoresmanual[,23,section,1,Inicialização,inicializacaomanual[,24,section,1,Aplicativo
ROSA,aplicativomanual[,25,subsection,2,Linguagem do
aplicativo,linguagemmanual[,26,subsection,2,Símbolos na
interface,interfacemanual[,27,subsection,2,Widgets,widgetsmanual[,28,subsection,2,Reinicair
ou debuggar,debugmanual[,29,section,1,Coniguração do software eletrônica
embarcada e eletrônica de superfície,configmanual[,30,subsection,2,Display
Top/Bottomside,displaymanual[,31,subsection,2,System
GUI,guimanual[,33,section,1,Configuração do roteador
WiFi,roteadormanual]]]]]]]]]]]]]]]]]}]{PDFs/UserGuide.pdf}

% ---------------------------------------------------------------------------
% ---------------------------------------------------------------------------
\chapter{Proposta de mestrado - André} 
\includepdf[pages=-,
addtotoc={4,section,1,Introdução,introandre[,5,section,1,Motivação,
motivacaoandre[,7,section,1,Objetivos,objetivoandre[,8,section,1,Metodologia e
Resultados Esperados,resultadosandre[,10,section,1,Tópicos Propostos para
Pesquisa,topicosandre[,11,subsection,2,Definição do Tipo de
Bateria,deftipoandre[,11,subsection,2,Modelagem Matemática das
Células,modandre[,12,subsection,2,Definição da Topologia de Conexão das Células,
topologiaandre[,12,subsection,2,Estudo de Carregamento,
carregamentoandre[,12,subsection,2,Estudo de Descarregamento,
descarregamentoandre[,13,subsection,2,Estudo de
Caso,casoandre[,14,section,1,Bibliograia
Proposta,bibandre[,17,section,1,Cronograma
Proposto,cronogramaandre[,18,section,1,Conclusão,conclusaoandre[,19,section,1,Referência
Bibliográficas,refbibandre]]]]]]]]]]]]]}]{PDFs/propostaAndre.pdf}
% --------------------------------------------------------------------------- 
% ---------------------------------------------------------------------------
\chapter{Artigo State of Charge Estimation and Battery Balancing Control -
André} 
\includepdf[pages=-,addtotoc={1,section,1,Introduction,intro2andre[,2,section,1,Battery
Cell Model,battery2andre[,3,section,1,Battery Cell Parameters Identification,
identandre[,3,section,1,Balancing Problem
Formulation,formulationandre[,4,section,1,Passive Balancing Control
Scheme,balancingandre[,5,section,1,SoC and Model Parameters
Estimator,socandre[,8,section,1,Conclusion,conclusion2andre[,8,section,1,References,ref2andre]]]]]]]}]{PDFs/artigoandre.pdf}

% ---------------------------------------------------------------------------
% ---------------------------------------------------------------------------
\chapter{Artigo CITENEL} 
\includepdf[pages=-,
addtotoc={1,section,1,Introdução,introcitanel[,2,section,1,Descrição
do Problema, problemacitenel[,3,section,1,Estado da
arte,artecitenel[,3,section,1,Descrição do
ROSA,rosacitenel[,4,section,1,Sistema embarcado
submarino,embarcadocitenel[,6,section,1,Sistema de
superfície,superficiecitenel[,7,section,1,Inspeção,inspecaocitenel[,10,section,1,Resultados,resultadoscitenel[,10,section,1,Benefícios,beneficioscitenel[,11,section,1,Conclusões,conclusaocitenel[,11,section,1,Referências
bibliogáficas,refbibcitenel]]]]]]]]]}]{PDFs/citenel.pdf}
% ---------------------------------------------------------------------------
% ---------------------------------------------------------------------------

\chapter{Artigo GDF Suez Innovation}
\includepdf[pages=-,
addtotoc={1,section,1,Introduction,introgdf[,2,section,1,Problematic,
problemagdf[,3,section,1,ROSA -
Robot,rosagdf[,4,section,1,Experimental tests and
results,resultadosgdf[,4,section,1,Conclusion
and future
work,conclusaogdf[,5,section,1,Team,teamgdf[,5,section,1,References,refgdf]]]]]}]{PDFs/GDF.pdf}
 
% ---------------------------------------------------------------------------
% ---------------------------------------------------------------------------

%\section{Proposta de mestrado Gabriel} 

%\section{Proposta de mestrado Elael}  

% ---------------------------------------------------------------------------
% ---------------------------------------------------------------------------
\chapter{Publicações e imprensa}
\includepdf[pages=-,addtotoc={1,section,1,Publicação no
Jornal de Jirau 01,jirau01pub}]{PDFs/01PubJirau.pdf}
\includepdf[pages=-,addtotoc={1,section,1,Publicação no
Jornal de Jirau 02,jirau02pub}]{PDFs/02PubJirau2.pdf}
\includepdf[pages=-,addtotoc={1,section,1,Publicação no
Planeta COPPE,coppepub}]{PDFs/03PubPlanetaCOPPE.pdf}
\includepdf[pages=1-3,addtotoc={1,section,1,Publicação no
Diário de Porto Velho,rondpub}]{PDFs/04PubRondonia.pdf}

\section{Vídeo conceito projeto ROSA}
Disponível em \href{https://youtu.be/Cw6f-8JNUK0}{Youtube}.


\section{Vídeo promocional projeto ROSA} 

% ---------------------------------------------------------------------------
% ---------------------------------------------------------------------------
%\chapter{Justificativa das alterações das informações relacionadas ao projeto
%base do XML inicial e a composição do XML final}
%\newcolumntype{C}[1]{>{\centering}m{#1}}

A Energia Sustentavel do Brasil (ESBR) optou por financiar como contrapartida no
projeto as rubricas que foram orçadas incialmente como custo da empresa
para realização do projeto, tais como Recursos Humanos (RH), Viagens Diárias
(VD), Material de Consumo (MC), Outros (OU), portanto todos os custos das
rubricas foram absorvidos pela empresa sendo retirados da conta da empresa e
não dos recursos do P&D. Portanto, todos os custos que a equipe da empresa teve
com o projeto foram custeadas pela ESBR.

Vale ressaltar ainda que, no arquivo do xml original constava como equipe da
executora do projeto a Fundação COPPETEC/UFRJ:

\begin{itemize}
\item Ramon Romankevicius Costa (CO);
\item Fernando Cesar Lizarralde (PE);
\item Sylvain Joyeux (PE);
\item Gustavo Medeiros Freitas (PE);
\item Thiago Toledo (PE);
\item Julia Campana (PE);
\item Rodrigo Fonseca Carneiro (PE);
\item Anderson Patury Sangreman (PE);
\item Sabrina Kunst (AT);
\item Joyce Mergulhão de Araújo (AT);

\end{itemize}


Porém, esta equipe foi alterada no decorrer do projeto, tendo em vista que 
o arquivo eletrônico XML do Projeto de P&D foi encaminhado para ANEEL em 27 de março de 2013 por meio do Sistema de Gestão de P&D/ANEEL, e o arquivo
eletrônico XML de início de execução do Projeto foi encaminhado para ANEEL em
08 de outubro de 2013 por meio do Sistema de Gestão de P&D/ANEEL. A
assinatura do Contrato entre a Energia Sustentável do Brasil S.A. (ESBR) e a
Fundação Coordenação de Projetos, Pesquisa e Estudos Tecnológicos –
COPPETEC/UFRJ foi em 08 de outubro de 2013. Porém, durante este intervalo
de tempo a equipe proposta originalmente no arquivo do Projeto encaminhado em
27 de março de 2013 a ANEEL teve alteração, pois alguns dos pesquisadores que
haviam sido selecionados para compor a equipe do projeto receberam propostas
de trabalho, portanto, houve a necessidade de uma nova seleção, mas somente
após a assinatura do contrato foi que a COPPETEC autorizou a contratação dos
demais integrantes para compor a equipe a qual contava somente com o
coordenador Ramon Costa e a pesquisadora Julia Campana. Devido ao novo
cenário conforme supracitado foi aberto um processo seletivo, coordenado pelo
Prof. Ramon Costa, mas que devido à dificuldade e escassez de disponibilidade
de interessados e, visando não comprometer o andamento e quiçá o sucesso da
dos estudos da pesquisa do referido projeto, o coordenador Ramon abriu o convite
para alguns alunos formados do curso de Engenharia e Controle e Automação da
Poli/UFRJ os quais haviam se inscrito para o mestrado no Programa de
Engenharia Elétrica da COPPE, sendo selecionados os pesquisadores para o
projeto: Renan Salles de Freitas, Gabriel Alcântara Costa Silva e Eduardo Elael
de Mello.
Posteriormente, durante a execução do projeto foi detectado a necessidade de um
suporte para a demanda burocrática do processo administrativo para o projeto que
até então não estava no escopo da proposta inicial encaminhada a ANEEL.
Portanto, um outro processo seletivo foi realizado para preencher o cargo de

Posteriormente, durante a execução do projeto foi detectado a necessidade de um
suporte para a demanda burocrática do processo administrativo para o projeto que
até então não estava no escopo da proposta inicial encaminhada a ANEEL.
Portanto, um outro processo seletivo foi realizado para preencher o cargo de

Assistente Administrativo sendo selecionada Alana Monteiro com admissão em 03
de março de 2013, respeitando-se o regimento interno da Fundação para seleção.

Portanto, a equipe que finalizou o projeto e que consta no xml final é a
seguinte:

\begin{itemize}
\item Ramon Romankevicius Costa (CO);
\item Gabriel Alcantara Costa Silva (PE);
\item Eduardo Elael de Melo Soares (PE);
\item Renan Sales de Freitas (PE);
\item Alana Monteiro Lima (AA);
\item Julia Ramos Campana (PE);
\item Andre Abido Figueiró (PE);
\item Alessandro Peixoto Jacoud (PE);
\end{itemize}

Importante deixar registrado que no âmbito dos projetos de Pesquisa e
Desenvolvimento, a Fundação COPPETEC pode fornecer um suporte ao projeto
para seleção de pessoal, entretanto, o critério para a seleção e a escolha da
equipe, segundo a política da Fundação, fica a cargo do coordenador do projeto.
As restrições são analisadas pela Comissão de Ética, que recomenda uma
composição da equipe, responsável pelo desenvolvimento do projeto
predominantemente envolvida com a Universidade, ou seja, é composta por
professores, alunos e técnicos da Universidade.

Segue alguns dos critérios que são adotados pelo regimento interno da Fundação
para a seleção:

Seleção de aluno de Iniciação Cientifica
\begin{itemize}
  \item Estar regulamente matriculado;
  \item Ter boa avalição acadêmica;
  \item Ter experiência (caso seja necessário para área de atuação) e;
  \item Ter noções de língua inglesa.
 \end{itemize}
  
Seleção de aluno Mestrado
\begin{itemize}
  \item Estar regulamente matriculado em um dos programas de pós-gradução;
  \item Ter boa avaliação acadêmica
  \item Ter possibilidade de alinhar seu tema na linha de pesquisa do projeto
contratante;
  \item Ter experiência com projetos de Pesquisa e Desenvolvimento e;
  \item Ter inglês fluente.
\end{itemize}

Seleção de Pesquisador 
\begin{itemize}
  \item Ter ensino superior reconhecido pelo MEC;
  \item Estar regularmente matriculado em um dos programas de pós-graduação;
  \item Ter experiência profissional na área do projeto;
  \item Dominar as ferramentas de trabalho (dependendo do escopo do projeto, as
habilidades dos candidatos podem ser avaliadas para garantir que possuam o
conhecimento necessário para a execução do projeto) e;
  \item Inglês fluente. 
\end{itemize} 

Seleção de Assistente Administrativo
\begin{itemize}
  \item Ter diploma de técnico em administração;
  \item Estar cursando Graduação em Administração;
  \item Ter experiência na área de Administração
  \item Dominar as ferramentas pertinentes ao proposto no escopo do projeto (as
habilidades dos candidatos podem ser avaliadas para garantir que possuam o
conhecimento necessário para a execução do projeto) e;
  \item Ter noções de inglês.
\end{itemize}

Aproveitamos todo esse contexto para mostrar que assim como todo o projeto tem
suas dificuldades em recrutar e até mesmo manter um recurso humano, no projeto
de P\&D não é muito diferente, enfrentamos todas as dificuldades principalmente
pela dificuldade em manter um pesquisador frente as várias ofertas de trabalhos
que os pesquisadores recebem de empresas''.

“Tendo em vista a necessidade de testes finais in loco com os equipamentos e
toda a eletrônica do projeto, aproveitou-se a oportunidade de parada de máquinas
na UHE JIRAU, para constatar a funcionalidade do robô, a fim de apresentar os
testes práticos e consolidação dos dados obtidos no laboratório de pesquisa até
então.
Porém, o período das paradas das Unidades Geradoras na UHE Jirau o qual
seriam aproveitados para realização dos testes, conforme supracitado, para a
certificação da funcionalidade do robô coincidiu com o feriado local da cidade do
Rio de Janeiro sede da COPPETEC/UFRJ, ou seja, as atividades de testes
compreenderam o período de 19/01/15 a 23/01/2015 sendo que o dia 20/01/15
terça-feira foi feriado de dia de São Sebastião Padroeiro da cidade do Rio de
Janeiro, no qual a segunda-feira dia 19/01/2015 foi ponto facultativo no período
classificado como feriado e recesso na cidade do Rio de Janeiro gerando assim
um pequeno excedente das horas, ocasionado justamente pelo feriado na cidade
do Rio de Janeiro. Importante ressaltar que até então não tínhamos autorização
para parada de máquina em outro período para realização do trabalho dos
pesquisadores.
Importante ressaltar que a UHE JIRAU (localizado na cidade de Porto Velho,
Estado de Rondônia) possui um índice de disponibilidade contratual de 99,5%
determinado no edital do leilão A-5 de concessão do empreendimento. Portanto,
devido a importância da contribuição da geração da UHE no Sistema Interligado
Nacional (SIN), e diante do supracitado não se podia parar a Unidade Geradora
sem autorização da Operador Nacional do Sistema e consequentemente da
ANEEL, pois qualquer indisponibilidade poderia acarretar em prejuízos no SIN.
Diante dos fatos expostos, não se tinha a previsão exata do próximo período de
parada de máquinas na UHE para que a equipe pudesse aproveitar a oportunidade
para realizar os testes do robô. E, a possibilidade de postergar a viagem poderia
trazer sérios prejuízos no desenvolvimento dos resultados da pesquisa podendo
até comprometê-la.
Vale ressaltar ainda que a Usina Hidrelétrica Jirau foi projetada para manter alta
performance de suas turbinas, nas diferentes vazões do Rio Madeira, ou seja,
opera normalmente tanto em períodos de seca, quanto em períodos de cheias
extraordinárias.

Mesmo durante a motorização da Usina, a UHE Jirau tem mantido seus índices
de disponibilidade contratual muito próximos do percentual de 99,5%
determinados no edital do leilão de concessão do empreendimento. O projeto de
construção civil, a tecnologia das turbinas adotadas pela ESBR e os estudos
aprofundados do comportamento do Rio Madeira, possibilitaram colocar, apenas
no ano de 2014, 22 Unidades Geradoras em operação. Um feito inédito para o
Brasil, que acrescentou 1.650 MW de potência ao SIN no ano passado.
Em 2016, ano de conclusão da Hidrelétrica, a Usina Jirau vai operar 50 UGs. Ao
todo, sua capacidade instalada de 3.750 MW, será suficiente para abastecer mais
de 10 milhões de residências por todo o país.
Aproveita-se a oportunidade para salientar que não houve nenhum tipo de
excedentes nos próximos meses de execução do projeto até seu término”.

Segue na planilha abaixo os demonstrativos dos custos do projeto, ou seja, os
valores previstos incialmente no xml inicial e os valores realizados no
projeto, bem como as devidas justificativas para as diferenças entre o previsto versus realizado.


\includepdf{PDFs/Tabela_Gastos.pdf}

\standardchapterstyle   
\end{document}