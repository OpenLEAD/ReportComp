A pesquisa e o desenvolvimento (P\&D) tem como propósito fomentar o avanço
tecnológico e novas maneiras de desenvolver um tipos específicos de conhecimento
no país. O desenvolvimento do robô ROSA, no âmbito P\&D é um exemplo de como a
parceria entre agências do governo e uni- versidades federais podem colaborar
para a capacitação tecnológica e o desenvolvimento de novas tecnologias.
Especificamente na área de robótica, o projeto ROSA mostra como otimização e
automação de diversos processos de trabalho pode contribuir na indústria
energética. Em termos acadêmicos, esta linha de projetos já originou três
dissertações de mestrado no Programa de Engenharia Elétrica (PEE), Universidade
Federal do Rio de Janeiro (UFRJ), abordando os seguintes tópicos:

\begin{itemize}
\item André Abido Figueiró, Estudo de Modelagem de Sistemas de Gerenciamento e
Balanceamento de Baterias.
\item Gabriel Silva Alcântara, \textit{Underwater 6 DOF Localization Using
Imaging Sonars}
\item Eduardo Elael, \textit{Underwater Mapping Using Imaging Sonar}
\end{itemize}

Os documentos comprobatórios das inscrições de mestrado seguem abaixo:

\includepdf{PDFs/abelha_mestrado}
\includepdf{PDFs/elael_mestrado}