%TODO
\chapter*{Contextualização}
Visando um aumento na segurança e otimização das operações realizadas na usina 
hidrelétrica de Jiray (3,750 MW Brasil), ESBR estabelceu uma parceria com a 
Universidade Federal do Rio de Janeiro para o desenvolvimento de um projeto de 
inovação de um robô submarino para monitoração e mapeamento (ROSA), para auxiliar
nas operações de manutenção de \textit{stoplogs}. O projeto ROSA teve início em
2013 e seu término ocorreu Fevereiro de 2015, totalizando um investimento de
R\$ 2.8 milhões.

\textit{Stoplogs} do barreiras metálicas modulares que são empilhadas na entrada
e sáida da tubina, permitindo o seu vedamento e a drenagem da região para
manutenção. Em Jirau, cada \textit{stoplog} pesa aproximadamente duas toneladas
e é necessário o empilhamento de 8 \textit{stoplogs} em cada extremidade da
turbina pra o seu selamento.

Um operador instala e remove os \textit{stoplogs}, utilizando um pórtico
rolante. Atualmente, a única informação disponível sobre o status da operação,
após a completa submersão do \textit{stoplog}, é a tensão do cabo. Essa operação
às cegas resulta, ocasionalmente, em falhas causadas por influência mecânicas ou
ambientais. Alguns exemplos são:
\begin{itemize}
  \item O acoplamento correto em somente um dos olhais do \textit{stoplog},
  resultando em uma movimentação inclinada e danificando a estrutura;
  \item Detritos e sedimentos acumulados na soleira, acarretando em um selamento
  imperfeito e, consequentemente, impossibilitando uma drenagem efetiva
\end{itemize}

Falhas na operação podem ser resolvidas somentras por meio da intervenção de
mergulhadores. As turbinas adjacentes em operação e o turbilhamento do rio podem
ocasionar em acidentes, tornando essa medida uma tarefa de alto risco à vida dos
mergulhadores.

\chapter*{Revisão Bibiliográfica}

No estudo de Lewin(1995) sobre as propriedades físicas e mecânicas de 
\textit{stoplogs} e viga pescadoras, o autor afirma em seu livro que o
\textit{stoplog} ideal possuira um sensor de contato e uma válvula reguladora da
pressão hidrostática, o que resolveria a maioria dos problemas apresentados
nesse tipo de operação. Entretanto, em Group (2006), Inglaterra, foi proposto
uma norma técnica para a produção de \textit{stoplogs} sem o sensoriamento
previamente citado, devido aos altos custos de implementação e manutenção. Em
2006, Hatch, uma consultoria de engenharia e desevolvimento, desenvolveu um
sistema comporto por uma viga pescadora instrumentada com sensores de
proximidade e servomotores elétricos submersíveis para atuar idependentemente em
cada garra. Em 2007, o sistema foi instalado na represa do Lago Ivanhoe,
reduzindo consideravelmente a necessidade de intervenção manual e aumentando,
assim, a segurança da operação (Hatch,2009).

Analogamente, a Atlas Polar automatiza o processo de instalação de
stoplogs, o sistema é consiste na utilização de um sensor indutivo para a
leitura do contato com o \textit{stoplog} e um sensor de força, do tipo
\textit{stain gauge}, para indicar a liberação ou levantamento de
\textit{stoplogs} (Polar,2015).

Nenhuma das soluções presentes na literatura são viáveis comercialmente para o
contexto da usina de Jirau. As soluções propostas requerem a aquisição de todo o
sistema composto por pórtico rolante e viga-pescadora para a integração dos
sensores de carga, em vez de se alterar a infraestrutura existente. Não foi
encontrado nenhum trabalho anterior referenciando a utilização de sonares para a
inspeção dos trilhos guia e da soleira da entrada e saída da turbina.

\chapter{Aspecto inovador do projeto}

O desenvolvimento de um sistema robótico submarino capaz de mapear e monitorar
operações com \textit{stoplogs}, resultando em um ganho econômico, por meio da
redução das paradas na produção, e aumento da segurança da operação,
reduzinho a necessidade da intervenção de mergulhadores.

\chapter{Descrição da inovação}

O robô de monitoramento e mapeamento submarino (ROSA) é projetado para evitar a
maioria dos problemas relacionados à operações com \textit{stoplogs}. O sistema
submarino realiza a monitoração em tempo real, processando e apresentando os
dados diretamente em um tablet instalado na cabine do operador. Algoritmos de
controle ''inteligentes'' auxiliam o operador no processo de tomada de decisão,
por meio de alarmes indicando falhas na operação.

O sistema consiste em uma eletrônica submarina embarcada desenvolvida
exclusivamente para o projeto, que se comunica com os sensores integrados à
viga. As medidas realizadas pelo sensor são transmitidas para a
eletrônica em terra através de um umbilical e, após o processamento, são
transmitida via Wi-Fi para o tablet com o operador.

O conjunto de sensores responsáveis pelo monitoramentos é constituído por
sensores de inclinação, profundidade e proximidade. Cada etapa do processo pesca
e retirada dos \textit{stoplogs} é, então, monitorado.

O sistema de mapeamento é constituído por um sensor sonar acoplado a um unidade
pan e tilt. A informação aquisitada pelo sonar é apresentada ao operador
sobreposta a um modelo 3D da estrutura submersa, possibilitando uma visualização
intuitiva dos sedimentos. As ondas sonoras emitidas e recebidas pelo sonar são
processadas probabilisticamente, e de acordo com a potência e tempo de resposta
do eco recebido, essas informações são inseridas em um mapa 3D de ocupância
probabilística e por meio de um filtro de partículas, o resultado mais provável
é estimado.

O projeto acrescenta uma inovação para o processo atual, no qual a operação é
realizado ''às cegas''. A solução é aplicável a qualquer usina que utilize um
sistema de \textit{stoplogs} baseado em pórtico rolante e viga-pescadora.


\chapter{Aplicabilidade}

O resultado final do projeto foi um protótipo consistindo de: uma unidade de
superfície à prova d'água, certificada com IP68, capaz de suportar umidade e
altas temperaturas e uma unidade submarina, certificada com IP69K, capaz de
suportar até 100 bar de pressão. O sistema está, atualmente, instalado e sendo
utilizado na usina de Jirau.

O protótipo foi demonstrado para os representantes das usinas de Jirau e Santo
Antônio em uma operação real com \textit{stoplogs}. A tarefa de monitoração foi
realizada com sucesso, indicando todos os passos de uma operação e emitindo
alarmes para as falhas detectadas. O mapeamento também foi bem sucedido,
detectando detritos maiores que 20cm. ESBR e Santo Antônio demostraram interesse
na continuação de projetos de inovação na mesma frente de pesquisa. O
equipamento de monitoração deve ter seu custo de implementação otimizado e
deverá ser instalado permanentemente em cada viga pescadora, totalizando, entre
ambas as usinas, 21 unidades. A unidade de mapeamento deve ter sua capacidade
extendida para poder mapear os danos causados pela cavitação nas paredes de
concreto.
 
\chapter{Relevância}
O ganho econômico gerado pelo sistema ROSA pode ser estimado pela quantidade de
falhas operacionais com \textit{stoplogs} que podem ser evitadas com a
utilização diária do sistema na usina de Jirau. Durante um período de 2 anos,
falhas envolvendo \textit{stoplogs} resultaram em uma perda de disponibilidade
de produção (Aumento da Energia Disponível) de 37 dias. Cada turbina
em Jirau produz 75MWh, considerando o preço da energia no mercado brasileiro de
aproximadamente R\$186,00 por MWh, a prevenção dessas falhas resultariam em uma
economia de R\$12.358.238,04 no período.

O ganho de segurança do trabalho com a utilização do robô rosa pode ser estimado
pela redução do número de intervenções com a utilização de mergulhadores
necessárias. O risco desse tipo de intervenção é classificado pela HSE como
elevado, uma vez que representa um risco real de perda de vidas humanas. Em
2014, foi registrado um incidente na usina de Jirau relacionado a uma
intervenção envolvendo a utilização de mergulhadores. Neste incidente, um
mergulhado foi arrastado pela correnteza da turbina adjacente, a corda de
segurança o segurou em um primeiro momento, porém o mesmo foi forçado a cortá-la
pois a pressão não o permitia respirar. O mergulhador for arrastado, então,
através do sistema rio abaixo, felizmente o acidente não acabou sendo fatal.

Problemas relacionados a \textit{stoplogs} representaram, no total, nos últimos
2 anos, em aproximadamente 25 dias de intervenções com mergulhadores. Desde
então, a perda de produção é relacionada ao tempo necessário em mobilizar os
mergulhadores, colocá-los na água e, finalmente, resolver o problema em si. O
robô ROSA é um sistema de monitoramento e inspeção, que tem como objetivo
em auxiliar o operador, afim de reduzir a ocorrência de falhas e evitar
completamente ações que resultem em um agravamento das condições de operações
após uma falha ter ocorrido. Considerando a lista de ocorrências envolvendo
\textit{stoplogs}, anexada a este documento, a utilização do robô ROSA
resultaria em uma redução de 75\% da exposição ao risco de acidentes. 

O projeto, apesar de curta duração (16 meses), resultou em 3 teses de 
mestrado, uma publicação técnica submetida ao CITENEL 2015, um artigo técnico 
em inglês submetido ao prêmio GDF Suez de inovação, e 6 publicações na imprensa: 
 


