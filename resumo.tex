%TODO
\section{Contextualização}
Visando um aumento na segurança e otimização das operações realizadas na usina 
hidrelétrica de Jirau (3,750 MW Brasil), a ESBR estabelceu uma parceria com a 
Universidade Federal do Rio de Janeiro para o desenvolvimento de um projeto de 
inovação de um robô submarino para monitoração e mapeamento (ROSA), com o
objetivo de auxiliar nas operações de manutenção de \textit{stoplogs}. O projeto
ROSA teve início em 2013 e seu término ocorreu em Fevereiro de 2015, totalizando
um investimento de R\$ 2.8 milhões.

\textit{Stoplogs} são barreiras metálicas modulares que são empilhadas na
entrada e sáida de cada tubina, permitindo o vedamento e a drenagem da
região para manutenção. Em Jirau, cada \textit{stoplog} pesa, aproximadamente,
duas toneladas e é necessário o empilhamento de 8 \texit{stoplogs} em cada extremidade
da turbina para o seu selamento.

Um operador instala e remove os \textir{stoplogs}, utilizando um pórtico
rolante. Atualmente, a única informação disponível sobre o status dessa
operação, após a completa submersão do \textit{stoplog}, é a tensão do cabo do
pórtico. Essa operação ''ás cegas'' resulta, ocasionalmente, em falhas causadas 
por influências mecânicas ou ambientais. Alguns exemplos são:

\begin{itemize}
  \item O acoplamento correto em somente um dos olhais do \textit{stoplog},
  resultando em uma movimentação inclinada e danificando a estrutura;
  \item Detritos e sedimentos acumulados na soleira, acarretando em um selamento
  imperfeito e, consequentemente, impossibilitando uma drenagem efetiva
\end{itemize}

Falhas na operação podem ser resolvidas somente por meio da intervenção de
mergulhadores. Entretanto,o fluxo das turbinas adjacentes em operação e o
turbilhamento do rio podem ocasionar em acidentes, tornando essa medida uma 
tarefa de alto risco à vida dos mergulhadores.

\section{Revisão Bibiliográfica}

No estudo de Lewin \cite{jack} sobre as propriedades físicas e mecânicas de 
\textit{stoplogs} e vigas pescadoras, o autor afirma, em seu livro, que o
\textit{stoplog} ideal possuiria um sensor de contato e uma válvula reguladora
de pressão hidrostática, o que resolveria a maioria dos problemas apresentados
nesse tipo de operação. Entretanto, em Group \cite{pinc}, Inglaterra, foi
propost uma norma técnica para a produção de \textit{stoplogs} sem o sensoriamento
previamente citado, devido aos altos custos de implementação e manutenção. Em
2006, Hatch, uma consultoria de engenharia e desevolvimento,
desenvolveu um sistema composto por uma viga pescadora instrumentada com sensores de
proximidade e servomotores elétricos submersíveis para atuar, idependentemente,
em cada garra. Em 2007, o sistema foi instalado na represa do Lago Ivanhoe,
reduzindo consideravelmente a necessidade de intervenção manual e aumentando,
assim, a segurança da operação (Hatch,2009)\cite{hatch}.

Em uma aplicação semelhante, a Atlas Polar automatizou o processo
de instalação e remoção de stoplogs, o sistema consiste na utilização de um sensor indutivo para a
leitura do contato com o \textit{stoplog} e um sensor de força, do tipo
\textit{stain gauge}, para indicar a liberação ou levantamento de
\textit{stoplogs} (Polar,2015)\cite{atlas}.

Porém, nenhuma das soluções presentes na literatura são viáveis comercialmente
para o contexto da usina de Jirau. As soluções propostas requerem a aquisição de todo o
sistema composto por pórtico rolante e viga-pescadora para a integração dos
sensores de carga, em vez da possibilidade de se alterar a infraestrutura já
existente.
Não foi encontrado, também, nenhum trabalho anterior referenciando a utilização
de sonares para a inspeção dos trilhos guia e da soleira da entrada e saída da turbina.

\section{Aspecto inovador do projeto}

O desenvolvimento de um sistema robótico submarino capaz de mapear e monitorar
operações com \textit{stoplogs}, resultando em um ganho econômico, por meio da
redução das paradas na produção e aumento da segurança da operação,
reduzinho a necessidade da intervenção de mergulhadores.

\section{Descrição da inovação}

O robô de monitoramento e mapeamento submarino (ROSA) é projetado para evitar a
maioria dos problemas relacionados à operações com \textit{stoplogs}. O sistema
submarino realiza o monitoramento em tempo real, processando e apresentando os
dados diretamente em um tablet instalado na cabine do operador. Algoritmos de
controle ''inteligentes'' auxiliam o operador no processo de tomada de decisão,
por meio de dados do estado do sistema e, também, alarmes indicando falhas na
operação.

O sistema consiste em uma eletrônica submarina embarcada, desenvolvida
exclusivamente para o projeto, que se comunica com os sensores integrados à
viga. As medidas realizadas pelo sensor são transmitidas para a
eletrônica em terra através de um umbilical e, após o processamento, são
transmitidas via Wi-Fi para o tablet com o operador.

O conjunto de sensores responsáveis pelo monitoramento é constituído por
sensores de inclinação, profundidade e proximidade. Cada etapa do processo de
pesca e retirada dos \textit{stoplogs} é, então, monitorado.

O sistema de mapeamento é constituído por uma eletrônica submarina e um sensor
sonar acoplado a uma unidade pan e tilt, que se comunicam com a eletrônica em
terra.
A informação aquisitada pelo sonar é apresentada ao operador, sobreposta a um 
modelo 3D da estrutura submersa, possibilitando, assim, uma
visualização e identificação intuitiva dos sedimentos acumulados.
As ondas sonoras emitidas e recebidas pelo sonar são processadas 
probabilisticamente, e de acordo com a potência e tempo de resposta
do eco recebido, essas informações são inseridas em um mapa 3D de ocupância
probabilística e por meio de um filtro de partículas, o resultado mais provável
é estimado.

O projeto acrescenta uma inovação para o processo atual, no qual a operação é
realizado ''às cegas''. A solução é aplicável a qualquer usina que utilize um
sistema de \textit{stoplogs} baseado em pórtico rolante e viga-pescadora.


\section{Aplicabilidade}

O resultado final do projeto foi um protótipo consistindo de: uma unidade de
superfície à prova d'água, certificada com IP68, capaz de suportar umidade e
altas temperaturas e uma unidade submarina, certificada com IP69K, capaz de
suportar até 100 bar de pressão. Atualmente, o sistema está instalado e sendo
utilizado na usina de Jirau.

O protótipo foi demonstrado para os representantes das usinas de Jirau e Santo
Antônio em uma operação real com \textit{stoplogs}. A tarefa de monitoramento
foi realizada com sucesso, indicando todos os passos de uma operação e emitindo
alarmes para as falhas detectadas. O mapeamento também foi bem sucedido,
detectando detritos maiores que 20cm. ESBR e Santo Antônio demostraram interesse
na continuação de projetos de inovação na mesma frente de pesquisa. O
equipamento de monitoração deve ter seu custo de implementação otimizado e
deverá ser instalado permanentemente em cada viga pescadora, totalizando, entre
ambas as usinas, 21 unidades. A unidade de mapeamento deve ter sua capacidade
extendida para poder mapear os danos causados pela cavitação nas paredes de
concreto.
 
\section{Relevância}

O ganho econômico gerado pelo sistema ROSA pode ser estimado pela quantidade de
falhas operacionais com \textit{stoplogs} que podem ser evitadas com a
utilização diária do sistema na usina de Jirau. Durante um período de 2 anos,
esse tipo de falha resultou em uma perda de
disponibilidade de produção (Aumento da Energia Disponível) de 37 dias.
Considerando que cada turbina em Jirau produz 75MWh e o preço da
energia no mercado brasileiro de aproximadamente R\$186,00 por MWh, a prevenção 
dessas falhas resultariam em uma economia de R\$12.358.238,04 no período.

O ganho de segurança do trabalho com a utilização do robô ROSA pode ser estimado
pela redução do número de intervenções necessárias com a utilização de
mergulhadores. O risco desse tipo de intervenção é classificado pela HSE como
elevado, uma vez que representa um risco real de perda de vidas humanas. Em
2014, foi registrado um incidente na usina de Jirau relacionado a uma
intervenção envolvendo a utilização de mergulhadores. Neste incidente, um
mergulhado foi arrastado pela correnteza da turbina adjacente e a corda de
segurança o segurou em um primeiro momento, porém o mesmo foi forçado a cortá-la
pois a pressão não o permitia respirar. O mergulhador for arrastado, então,
através do sistema rio abaixo, felizmente o acidente não acabou sendo fatal.

Problemas relacionados a \textit{stoplogs} representaram, no total, nos últimos
2 anos, em aproximadamente 25 dias de intervenções com mergulhadores. A perda de 
produção é relacionada ao tempo necessário em mobilizar os
mergulhadores, colocá-los na água e, finalmente, resolver o problema em si. O
robô ROSA é um sistema de monitoramento e inspeção, que tem como objetivo
em auxiliar o operador, afim de reduzir a ocorrência de falhas e evitar
completamente ações que resultem em um agravamento das condições de operações
após uma falha ter ocorrido. Considerando a lista de ocorrências 
envolvendo \textit{stoplogs} abaixo, a utilização do robô ROSA
resultaria em uma redução de 75\% da exposição ao risco de acidentes. 

Falhas:
\begin{enumerate}
  \item Não ensecamento de UG devido a detritos na soleira e guiamento lateral:
  		\begin{itemize}
  		  \item UG 01, outubro/2014, 4 dias
  		  \item UG 03, outubro/2014, 4 dias
  		  \item UG 05, setembro/2014, 4 dias
  		  \item UG 32, setembro/2014. 4 dias	
  		\end{itemize}
  		OBS: Por consequência, todas as paradas de máquina a partir de
  		outubro de 2014, tem tido, necessariamente, mergulho
  		preventivos. 
  \item Queda de painel devido a “pesca” equivocada:
  		\begin{itemize}
  		  \item UG 32 (antes do enchimento), fevereiro/2013; 10 dias (recuperação
  		  soleira e guia lateral)
  		  \item Vertedouro Principal vão 5, dezembro 2013. 4 dias (tentativa de
  		  localização com sonar pois o mesmo saiu do vão)
  		\end{itemize}
  \item Não atuação de cabo frouxo ocasionando que o cabo do Pórtico se
  emaranhasse no painel e rompesse na tentativa de içamento. 7 dias de atraso 
  para tirar e 90 dias para chegada de novo cabo. Levando em conta a resolução 
  de todos os problemas, o pórtico ficará limitado por mais de 100 dias.
\end{enumerate}

O projeto, apesar de curta duração (16 meses), resultou em 3 teses de 
mestrado, uma publicação técnica submetida ao CITENEL 2015, um artigo técnico 
em inglês submetido ao prêmio GDF Suez de inovação, e 6 publicações na imprensa: 

%TODO Publicaçõe de imprensa
 
     
\section{Razoabilidade dos Custos}
O projeto ROSA custou R\$2.880.000,00 em sua fase de cadeia de pesquisa
aplicada. Ao final do projeto foi constatado que serão necessário 11 unidades 
do ROSA, cada um instalado permanentemente em uma viga pescadora, para a
utilização eficiente do sistema de monitoramento. O custo do ROSA, após a
otimização dos componentes e fabricação, é esperado ser de R\$35.000,00 por
unidade. Logo, o custo de implantação da solução seria de R\$385.000,00,
totalizando um custo total de R\$3.265.000,00 entre conceito e implantação.

O ganho econômico (aumento de energia disponível), que o ROSA teria
proporcionado se em operação nos últimos 2 anos em Jirau, é de R\$
12.358.238,04, como detalhado na secção anterior. Portanto, o projeto teria uma
expectativa de retorno de investimento de 6 meses.

\section{Transferência e difusão tecnológica dos resultados do projeto}
Um workshop de transferência de conhecimento foi realizado na UHE Jirau com os
Engenheiros de operação. Além, foi provido treinamento para manutenção e uso do 
protótipo, incluindo manual de uso. Ao total, um engenheiro de operação foi 
treinado na manutenção do software, dois técnicos instrumentistas foram treinados 
na instalação e uso do equipamento e dois operadores de pórtico rolante no uso 
do equipamento.

\section{Metodologia}

Após a análise e compreensão do problema de inserção e remoção de
\textit{stoplogs}, foi feita uma pesquisa bibliográfica e brainstorm com o
objetivo de alcançar um conceito sólido de solução ao problema. A partir do
resultado dessa pesquisa, foi desenvolvido um conceito base de solução robótica. 
Baseado neste conceito, foram realizadas pesquisas de tecnologias e de fornecedores 
de forma recursiva e convergente com relação aos resultados. Isto é, com base
nas pesquisas de solução tecnológicas possíveis, buscam-se fornecedores
compatíveis e, com o resultado e informação dos produtos dos fornecedores
encontrados, faz-se novamente uma pesquisa de tecnologia , agora mais
aprofundada, e assim sucessivamente, até encontrar-se um resultado final
satisfatório.

O escopo inicial de solução é, então, atualizado e detalhado de acordo com o 
resultado desta pesquisa, resultando no design do robô a ser construído no projeto. 
Após a especificação do design, os componentes necessários para construção são
adquiridos e manufaturados. Os sensores e componentes eletrônicos são testados
em campo para averiguar que funcionam como especificado e planejado no
projeto. Em paralelo, o software e interfaces necessários para o projeto são 
detalhados e implementados.  

Após a etapa de detalhamento, o projeto entra na etapa de execução, na qual
ocorre a integração do diversos componentes que constituem o robô.
O software necessário para o processamento dos dados é desenvolvido e o
sistema, como um conjunto, é testado em campo. 

A viagem de campo fornece informações sobre as falhas, limitações e melhorias 
necessárias ao sistema. O projeto entra, então, na etapa de encerramento. Nesta 
etapa, o sistema é aprimorado e testado recursivamente e o conhecimento de uso e 
operação do mesmo é transferido. Por fim, a última etapa do projet consiste na
entrega do equipamento, manual de uso e documentação de encerramento do projeto.

\bibliographystyle{ieeetr}
\bibliography{resumo}