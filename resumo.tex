%TODO
\chapter*{Contextualização}
Visando um aumento na segurança e otimização das operações realizadas na usina 
hidrelétrica de Jiray (3,750 MW Brasil), ESBR estabelceu uma parceria com a 
Universidade Federal do Rio de Janeiro para o desenvolvimento de um projeto de 
inovação de um robô submarino para monitoração e mapeamento (ROSA), para auxiliar
nas operações de manutenção de \textit{stoplogs}. O projeto ROSA teve início em
2013 e seu término ocorreu Fevereiro de 2015, totalizando um investimento de
R\$ 2.8 milhões.

\textit{Stoplogs} do barreiras metálicas modulares que são empilhadas na entrada
e sáida da tubina, permitindo o seu vedamento e a drenagem da região para
manutenção. Em Jirau, cada \textit{stoplog} pesa aproximadamente duas toneladas
e é necessário o empilhamento de 8 \texit{stoplogs} em cada extremidade da turbina
pra o seu selamento.

Um operador instala e remove os \textir{stoplogs}, utilizando um pórtico
rolante. Atualmente, a única informação disponível sobre o status da operação,
aṕos a completa submersão do \textit{stoplog}, é a tensão do cabo. Essa operação
''ás cegas'' resulta, ocasionalmente, em falhas causadas por influências
mecânicas ou ambientais. Alguns exemplos são:

\begin{itemize}
  \item O acoplamento correto em somente um dos olhais do \textir{stoplog},
  resultando em uma movimentação inclinada e danificando a estrutura;
  \item Detritos e sedimentos acumulados na soleira, acarretando em um selamento
  imperfeito e, consequentemente, impossibilitando uma drenagem efetiva
\end{itemize}

Falhas na operação podem ser resolvidas somentras por meio da intervenção de
mergulhadores. As turbinas adjacentes em operação e o turbilhamento do rio podem
ocasionar em acidentes, tornando essa medida uma tarefa de alto risco à vida dos
mergulhadores.

\chapter*{Revisão Bibiliográfica}

No estudo de Lewin(1995) sobre as propriedades físicas e mecânicas de 
\textit{stoplogs} e viga pescadoras, o autor afirma em seu livro que o
\textit{stoplog} ideal possuira um sensor de contato e uma válvula reguladora da
pressão hidrostática, o que resolveria a maioria dos problemas apresentados
nesse tipo de operação. Entretanto, em Group (2006), Inglaterra, foi proposto
uma norma técnica para a produção de \textit{stoplogs} sem o sensoriamento
previamente citado, devido aos altos custos de implementação e manutenção. Em
2006, Hatch, uma consultoria de engenharia e desevolvimento, desenvolveu um
sistema comporto por uma viga pescadora instrumentada com sensores de
proximidade e servomotores elétricos submersíveis para atuar idependentemente em
cada garra. Em 2007, o sistema foi instalado na represa do Lago Ivanhoe,
reduzindo consideravelmente a necessidade de intervenção manual e aumentando,
assim, a segurança da operação (Hatch,2009).

Analogamente, a Atlas Polar automatiza o processo de instalação de
stoplogs, o sistema é consiste na utilização de um sensor indutivo para a
leitura do contato com o \textit{stoplog} e um sensor de força, do tipo
\textit{stain gauge}, para indicar a liberação ou levantamento de
\textit{stoplogs} (Polar,2015).

Nenhuma das soluções presentes na literatura são viáveis comercialmente para o
contexto da usina de Jirau. As soluções propostas requerem a aquisição de todo o
sistema composto por pórtico rolante e viga-pescadora para a integração dos
sensores de carga, em vez de se alterar a infraestrutura existente. Não foi
encontrado nenhum trabalho anterior referenciando a utilização de sonares para a
inspeção dos trilhos guia e da soleira da entrada e saída da turbina.

\chapter{Aspecto inovador do projeto}

O desenvolvimento de um sistema robótico submarino capaz de mapear e monitorar
operações com \textit{stoplogs}, resultando em um ganho econômico, por meio da
redução das paradas na produção, e aumento da segurança da operação,
reduzinho a necessidade da intervenção de mergulhadores.

\chapter{Descrição da inovação}

O robô de monitoramento e mapeamento submarino (ROSA) é projetado para evitar a
maioria dos problemas relacionados à operações com \textit{stoplogs}. O sistema
submarino realiza a monitoração em tempo real, processando e apresentando os
dados diretamente em um tablet instalado na cabine do operador. Algoritmos de
controle ''inteligentes'' auxiliam o operador no processo de tomada de decisão,
por meio de alarmes indicando falhas na operação.

O sistema consiste em uma eletrônica submarina embarcada desenvolvida
exclusivamente para o projeto, que se comunica com os sensores integrados à
viga. As medidas realizadas pelo sensor são transmitidas para a
eletrônica em terra através de um umbilical e, após o processamento, são
transmitida via Wi-Fi para o tablet com o operador.

O conjunto de sensores responsáveis pelo monitoramentos é constituído por
sensores de inclinação, profundidade e proximidade. Cada etapa do processo pesca
e retirada dos \textit{stoplogs} é, então, monitorado.

O sistema de mapeamento é constituído por um sensor sonar acoplado a um unidade
pan e tilt. A informação aquisitada pelo sonar é apresentada ao operador
sobreposta a um modelo 3D da estrutura submersa, possibilitando uma visualização
intuitiva dos sedimentos.

%TODO %%%%%%%%%%%%%%%%%%%%%%%%%%%%%%%%%%%%%%%%%%%%%%%%%%%%%%%%%%%%%
  Multiple sound waves emitted by the sonar are modeled probabilistically in 
 3D according to the potency, the eco response is than store in a 3D probabilistic 
 grid map and finally fused through a particle filter to estimate the most probable map.
%%%%%%%%%%%%%%%%%%%%%%%%%%%%%%%%%%%%%%%%%%%%%%%%%%%%%%%%%%%%%%%%%%%%%%%%%

O projeto acrescenta uma inovação para o processo atual, no qual a operação é
realizado ''às cegas''. A solução é aplicável a qualquer usina que utilize um
sistema de \textit{stoplogs} baseado em pórtico rolante e viga-pescadora.


\chapter{Aplicabilidade}

The end result is a prototype: water tight surface unit with IP68 rating level, 
capable of withstanding high temperature and humidity, a IP69K rating subsea unit 
capable of withstanding up to 100 bar of pressure. The system is currently 
deployed and in use at Jirau.
	The prototype was demonstrated to Jirau and Santo Antonio in real Stoplog 
	operation. The monitoring payload worked as designed signaling all operation 
	steps, raising alarms for detected operational failures. The mapping payload 
	also worked as intended with a detection threshold for debris larger than 20cm. 
	ESBR and Santo Antonio showed interested in the continuation of the innovation 
	chain of the project. The monitoring payload is to be optimized cost wise 
	and installed permanently one in each Viga Pescadora, totalizing between 
	both power plants 21 Units. The mapping payload is to be extended in capability 
	to be able to map cavitation of the concrete walls. 
\chapter{Relevância}
The economical gain of the ROSA system can be estimated by the amount of Stoplog 
operational issues that would have been avoided in case the system was in daily 
use in Jirau powerplant. In Jirau, over a period of 2 years, stoplog faults resulted 
in a production availability loss (Aumento da Energia Disponível) of 37 days.
 Each turbine in Jirau produces 75 MWh, considering, the energy price in the free 
 market in Brazil of approximately 186,00 per MWh this would signify a possible 
 extra revenue of R\$12.358.238,04. 
	The gain in work safety of the ROSA can be estimated by the reduction of number 
	of interventional dives. The level of risk of this task in HSE is classified as 
	Major, since it represent a real risk of lost of a human life. The following 
	incident took place with an interventional dive in Jirau, 2014. A diver was 
	dragged to an adjacent inlet, the security rope held him initially, but the 
	water pressure was too high making breathing and moving back impossible. He 
	had to cut the rope and allow himself to be dragged through the system 
downriver. Luckily, the accident was not fatal. 
	Stoplog issues represented approximately 25 days of interventional dives 
	this past 2 years. Since, the downtime in production is related to the time 
	necessary to mobilize the divers, dive and resolve the issues. The ROSA system
	 is a monitoring and inspection system, helping the operator to reduce the 
	 occurrence of faults and completely avoid aggravating actions of operating 
	 under faulty condition. Considering the list of stop log occurrences, 
	 detailed in the attached document, the use of the ROSA would have resulted 
	 in a reduction of 75% in exposure to risk. 
	O projeto, apesar de curta duracao (16 meses), resultou em 3 teses de 
	mestrado, uma publicação técnica submetida ao CITENEL 2015, um artigo técnico 
	em inglês submetido ao prêmio GDF Suez de inovação, e 6 publicações na imprensa: 
 


