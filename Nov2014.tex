\section{Relatório de Viagem Jirau - Novembro 2014}
Data: 03 a 07 de Novembro de 2014.

Equipe: Eduardo Elael, Gabriel Alcântara, Julia Campana, Renan Salles e Rodrigo Carneiro.


Viagem de equipe do Projeto ROSA a UHE Jirau, realizada entre
os dia 3 e 7 de Novembro de 2014. O objetivo foi realizar testes de campo dos
componentes do Robô ROSA assim como os testes de usabilidade do aplicativo ROSA
- sistema que monitora as operações de inserção e remoção de stoplogs.

Como resultado dos testes de componentes, conclui-se que a posição usada na
garra ajustada de acordo com dados da viagem anterior funcionou com
perfeitamente, assim como  os sensores indutivos que também se comportaram da
forma esperada nos fornecendo dados necessários.  Também foram realizados
testes do Sonar modelo Seaking funcionou de acordo e nos forneceu dados a serem
processados e futramente alinhados com o sistema do robô.

No quesito usabilidade tivemos testes importantes com o operador da viga, que
utilizou pela primeira vez o sistema durante a operação, foram aplicados
testes de usabilidade, heurísticos e também uma entrevista para feedback
informal com realação a familiaridade e linguagem visual.

As figuras~\ref{nov20141}, ~\ref{nov20142}, ~\ref{nov20143}, ~\ref{nov20144},
~\ref{nov20145} foram obtidas durante a viagem de Novembro, 2014.


\begin{figure}[h!]
\centering
  \includegraphics[width=1\linewidth]{Fotos/Novembro2014/1.JPG}
  \caption{Montagem de sensores na garra pesacdora.}
  \label{nov20141}
\end{figure}

\begin{figure}[h!]
  \centering
  \includegraphics[width=1\linewidth]{Fotos/Novembro2014/3.JPG}
  \caption{Eletrônica embarcada acoplada a viga pescadora.}
  \label{nov20142}
\end{figure}

\begin{figure}[h!]
  \centering
  \includegraphics[width=1\linewidth]{Fotos/Novembro2014/4.JPG}
  \caption{Engenheiro fazendo ajustes de componentes.}
  \label{nov20143}
\end{figure}

\begin{figure}[h!]
  \centering
  \includegraphics[width=1\linewidth]{Fotos/Novembro2014/6.JPG}
  \caption{Sensor indutivo acoplado a garra pescadora.}
  \label{nov20144}
\end{figure}

\begin{figure}[h!]
  \centering
  \includegraphics[width=1\linewidth]{Fotos/Novembro2014/10.JPG}
  \caption{Operador em ação operando a viga pescadora.}
  \label{nov20145}
\end{figure}