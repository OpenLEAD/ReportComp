O sistema robótico desenvolvido requer uma pequena modificação permanente na
viga pescadora: o acoplamento de sensores indutivos nas garras em locais
estratégicos para a percepção de proximidade das garras com os olhais do
stoplog, garantindo o monitoramento da ocorrên- cia de pega ou liberação dos
stoplogs. Essa pequena alteração não prejudica em nada a operação e, em
conjunto com o sistema, auxilia no processo, como foi elucidado no
capítulo~\ref{chap::melpro}. Vale ressaltar que esta é uma melhoria
introduzida a um equipamento frequentemente utilizado na usina, pois os
sensores indutivos podem ser futuramente utilizados com outras eletrônicas e/ou
até com sistemas mais simples que visam utilizar parcialmente as
funcionalidades do sistema robótico ROSA desenvolvido. Tornando a viga
pescadora em uma viga inteligente.
