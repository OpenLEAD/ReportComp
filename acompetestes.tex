A abordagem metodológica do projeto teve início na elaboração da solução do problema. Após o
detalhamento do conceito base, foram planejadas as fases de desenvolvimento e execução que
abrangeram um total de 4 testes ao longo do projeto. Estes testes foram executados em campo na
UHE Jirau, por toda a equipe de pesquisadores do projeto durante um período de 5
dias a cada viagem realizada.

\emph{Teste 1:}
Realizado entre os 02/06/2015 a 06/06/2015 o primeiro teste de campo teve como objetivo verifi-
car a estrutura necessária para cabeamento e instalação do robô com seus dispositivos na viga
pescadora. Foi realizada a montagem, com treinamento do pessoal da UHE Jirau, e
a validação da funcionalidade do protótipo.

\emph{Teste 2:}
Realizado entre os 03/11/2014 a 07/11/2014 o segundo teste de campo teve como objetivo testar
os dispositivos do robô individualmente. Foram realizados testes com o Sonar
modelo SeaKing da empresa Tritech na soleira disponível, testes de sensores
indutivos nas garras da viga pescadora, e testes heurísticos de usabilidade
com a adição de sons no aplicativo dentro da cabine do operador.

\emph{Teste 3:}
Realizado entre os 19/01/2015 a 23/01/2015 o terceiro teste de campo teve como objetivo testar
o funcionamento do robô em sua totalidade. Foram realizados testes do sistema com todos seus
dispositivos ativos, bem como o teste heurístico e acompanhamento de atividades de dentro da
cabine do operador.

\emph{Teste 4:}
Realizado entre os 23/02/2015 a 27/02/2015 o último teste de campo teve como objetivo testar
todos os sistemas do robô ROSA, treinamento de usabilidade e amostragem final do projeto.