\newcolumntype{C}[1]{>{\centering}m{#1}}

A Energia Sustentavel do Brasil (ESBR) optou por financiar como contrapartida no
projeto as rubricas que foram orçadas incialmente como custo da empresa
para realização do projeto, tais como Recursos Humanos (RH), Viagens Diárias
(VD), Material de Consumo (MC), Outros (OU), portanto todos os custos das
rubricas foram absorvidos pela empresa sendo retirados da conta da empresa e
não dos recursos do P&D. Portanto, todos os custos que a equipe da empresa teve
com o projeto foram custeadas pela ESBR.

Vale ressaltar ainda que, no arquivo do xml original constava como equipe da
executora do projeto a Fundação COPPETEC/UFRJ:

\begin{itemize}
\item Ramon Romankevicius Costa (CO);
\item Fernando Cesar Lizarralde (PE);
\item Sylvain Joyeux (PE);
\item Gustavo Medeiros Freitas (PE);
\item Thiago Toledo (PE);
\item Julia Campana (PE);
\item Rodrigo Fonseca Carneiro (PE);
\item Anderson Patury Sangreman (PE);
\item Sabrina Kunst (AT);
\item Joyce Mergulhão de Araújo (AT);

\end{itemize}


Porém, esta equipe foi alterada no decorrer do projeto, tendo em vista que 
o arquivo eletrônico XML do Projeto de P&D foi encaminhado para ANEEL em 27 de março de 2013 por meio do Sistema de Gestão de P&D/ANEEL, e o arquivo
eletrônico XML de início de execução do Projeto foi encaminhado para ANEEL em
08 de outubro de 2013 por meio do Sistema de Gestão de P&D/ANEEL. A
assinatura do Contrato entre a Energia Sustentável do Brasil S.A. (ESBR) e a
Fundação Coordenação de Projetos, Pesquisa e Estudos Tecnológicos –
COPPETEC/UFRJ foi em 08 de outubro de 2013. Porém, durante este intervalo
de tempo a equipe proposta originalmente no arquivo do Projeto encaminhado em
27 de março de 2013 a ANEEL teve alteração, pois alguns dos pesquisadores que
haviam sido selecionados para compor a equipe do projeto receberam propostas
de trabalho, portanto, houve a necessidade de uma nova seleção, mas somente
após a assinatura do contrato foi que a COPPETEC autorizou a contratação dos
demais integrantes para compor a equipe a qual contava somente com o
coordenador Ramon Costa e a pesquisadora Julia Campana. Devido ao novo
cenário conforme supracitado foi aberto um processo seletivo, coordenado pelo
Prof. Ramon Costa, mas que devido à dificuldade e escassez de disponibilidade
de interessados e, visando não comprometer o andamento e quiçá o sucesso da
dos estudos da pesquisa do referido projeto, o coordenador Ramon abriu o convite
para alguns alunos formados do curso de Engenharia e Controle e Automação da
Poli/UFRJ os quais haviam se inscrito para o mestrado no Programa de
Engenharia Elétrica da COPPE, sendo selecionados os pesquisadores para o
projeto: Renan Salles de Freitas, Gabriel Alcântara Costa Silva e Eduardo Elael
de Mello.
Posteriormente, durante a execução do projeto foi detectado a necessidade de um
suporte para a demanda burocrática do processo administrativo para o projeto que
até então não estava no escopo da proposta inicial encaminhada a ANEEL.
Portanto, um outro processo seletivo foi realizado para preencher o cargo de

Posteriormente, durante a execução do projeto foi detectado a necessidade de um
suporte para a demanda burocrática do processo administrativo para o projeto que
até então não estava no escopo da proposta inicial encaminhada a ANEEL.
Portanto, um outro processo seletivo foi realizado para preencher o cargo de

Assistente Administrativo sendo selecionada Alana Monteiro com admissão em 03
de março de 2013, respeitando-se o regimento interno da Fundação para seleção.

Portanto, a equipe que finalizou o projeto e que consta no xml final é a
seguinte:

\begin{itemize}
\item Ramon Romankevicius Costa (CO);
\item Gabriel Alcantara Costa Silva (PE);
\item Eduardo Elael de Melo Soares (PE);
\item Renan Sales de Freitas (PE);
\item Alana Monteiro Lima (AA);
\item Julia Ramos Campana (PE);
\item Andre Abido Figueiró (PE);
\item Alessandro Peixoto Jacoud (PE);
\end{itemize}

Importante deixar registrado que no âmbito dos projetos de Pesquisa e
Desenvolvimento, a Fundação COPPETEC pode fornecer um suporte ao projeto
para seleção de pessoal, entretanto, o critério para a seleção e a escolha da
equipe, segundo a política da Fundação, fica a cargo do coordenador do projeto.
As restrições são analisadas pela Comissão de Ética, que recomenda uma
composição da equipe, responsável pelo desenvolvimento do projeto
predominantemente envolvida com a Universidade, ou seja, é composta por
professores, alunos e técnicos da Universidade.

Segue alguns dos critérios que são adotados pelo regimento interno da Fundação
para a seleção:

Seleção de aluno de Iniciação Cientifica
\begin{itemize}
  \item Estar regulamente matriculado;
  \item Ter boa avalição acadêmica;
  \item Ter experiência (caso seja necessário para área de atuação) e;
  \item Ter noções de língua inglesa.
 \end{itemize}
  
Seleção de aluno Mestrado
\begin{itemize}
  \item Estar regulamente matriculado em um dos programas de pós-gradução;
  \item Ter boa avaliação acadêmica
  \item Ter possibilidade de alinhar seu tema na linha de pesquisa do projeto
contratante;
  \item Ter experiência com projetos de Pesquisa e Desenvolvimento e;
  \item Ter inglês fluente.
\end{itemize}

Seleção de Pesquisador 
\begin{itemize}
  \item Ter ensino superior reconhecido pelo MEC;
  \item Estar regularmente matriculado em um dos programas de pós-graduação;
  \item Ter experiência profissional na área do projeto;
  \item Dominar as ferramentas de trabalho (dependendo do escopo do projeto, as
habilidades dos candidatos podem ser avaliadas para garantir que possuam o
conhecimento necessário para a execução do projeto) e;
  \item Inglês fluente. 
\end{itemize} 

Seleção de Assistente Administrativo
\begin{itemize}
  \item Ter diploma de técnico em administração;
  \item Estar cursando Graduação em Administração;
  \item Ter experiência na área de Administração
  \item Dominar as ferramentas pertinentes ao proposto no escopo do projeto (as
habilidades dos candidatos podem ser avaliadas para garantir que possuam o
conhecimento necessário para a execução do projeto) e;
  \item Ter noções de inglês.
\end{itemize}

Aproveitamos todo esse contexto para mostrar que assim como todo o projeto tem
suas dificuldades em recrutar e até mesmo manter um recurso humano, no projeto
de P\&D não é muito diferente, enfrentamos todas as dificuldades principalmente
pela dificuldade em manter um pesquisador frente as várias ofertas de trabalhos
que os pesquisadores recebem de empresas''.

“Tendo em vista a necessidade de testes finais in loco com os equipamentos e
toda a eletrônica do projeto, aproveitou-se a oportunidade de parada de máquinas
na UHE JIRAU, para constatar a funcionalidade do robô, a fim de apresentar os
testes práticos e consolidação dos dados obtidos no laboratório de pesquisa até
então.
Porém, o período das paradas das Unidades Geradoras na UHE Jirau o qual
seriam aproveitados para realização dos testes, conforme supracitado, para a
certificação da funcionalidade do robô coincidiu com o feriado local da cidade do
Rio de Janeiro sede da COPPETEC/UFRJ, ou seja, as atividades de testes
compreenderam o período de 19/01/15 a 23/01/2015 sendo que o dia 20/01/15
terça-feira foi feriado de dia de São Sebastião Padroeiro da cidade do Rio de
Janeiro, no qual a segunda-feira dia 19/01/2015 foi ponto facultativo no período
classificado como feriado e recesso na cidade do Rio de Janeiro gerando assim
um pequeno excedente das horas, ocasionado justamente pelo feriado na cidade
do Rio de Janeiro. Importante ressaltar que até então não tínhamos autorização
para parada de máquina em outro período para realização do trabalho dos
pesquisadores.
Importante ressaltar que a UHE JIRAU (localizado na cidade de Porto Velho,
Estado de Rondônia) possui um índice de disponibilidade contratual de 99,5%
determinado no edital do leilão A-5 de concessão do empreendimento. Portanto,
devido a importância da contribuição da geração da UHE no Sistema Interligado
Nacional (SIN), e diante do supracitado não se podia parar a Unidade Geradora
sem autorização da Operador Nacional do Sistema e consequentemente da
ANEEL, pois qualquer indisponibilidade poderia acarretar em prejuízos no SIN.
Diante dos fatos expostos, não se tinha a previsão exata do próximo período de
parada de máquinas na UHE para que a equipe pudesse aproveitar a oportunidade
para realizar os testes do robô. E, a possibilidade de postergar a viagem poderia
trazer sérios prejuízos no desenvolvimento dos resultados da pesquisa podendo
até comprometê-la.
Vale ressaltar ainda que a Usina Hidrelétrica Jirau foi projetada para manter alta
performance de suas turbinas, nas diferentes vazões do Rio Madeira, ou seja,
opera normalmente tanto em períodos de seca, quanto em períodos de cheias
extraordinárias.

Mesmo durante a motorização da Usina, a UHE Jirau tem mantido seus índices
de disponibilidade contratual muito próximos do percentual de 99,5%
determinados no edital do leilão de concessão do empreendimento. O projeto de
construção civil, a tecnologia das turbinas adotadas pela ESBR e os estudos
aprofundados do comportamento do Rio Madeira, possibilitaram colocar, apenas
no ano de 2014, 22 Unidades Geradoras em operação. Um feito inédito para o
Brasil, que acrescentou 1.650 MW de potência ao SIN no ano passado.
Em 2016, ano de conclusão da Hidrelétrica, a Usina Jirau vai operar 50 UGs. Ao
todo, sua capacidade instalada de 3.750 MW, será suficiente para abastecer mais
de 10 milhões de residências por todo o país.
Aproveita-se a oportunidade para salientar que não houve nenhum tipo de
excedentes nos próximos meses de execução do projeto até seu término”.

Segue na planilha abaixo os demonstrativos dos custos do projeto, ou seja, os
valores previstos incialmente no xml inicial e os valores realizados no
projeto, bem como as devidas justificativas para as diferenças entre o previsto versus realizado.


\includepdf{PDFs/Tabela_Gastos.pdf}