\newcolumntype{C}[1]{>{\centering}m{#1}}

Equipe empresa ESBR:
Na rubrica RH - A empresa optou por financiar todos os custos da equipe da
empresa ESBR.

Equipe empresa COPPETEC:
Os profissionais Alana Monteiro, Eduardo Elael, Gabriel Alcântara e Renan
Salles, os quais compõem a folha de apontamento de horas da Instituição
Executora, não estão listados no arquivo XML do projeto.

A justificativa da auditada para a exceção identificada é a seguinte:
"O arquivo eletrônico XML do Projeto de P\&D foi encaminhado para ANEEL em
27 de março de 2013 por meio do Sistema de Gestão de P\&D/ANEEL, e o arquivo
eletrônico XML de início de execução do Projeto foi encaminhado para ANEEL em
08 de outubro de 2013 por meio do Sistema de Gestão de P\&D/ANEEL. A
assinatura do Contrato entre a Energia Sustentável do Brasil S.A. (ESBR) e a
Fundação Coordenação de Projetos, Pesquisa e Estudos Tecnológicos –
COPPETEC/UFRJ foi em 08 de outubro de 2013. Porém, durante este intervalo
de tempo a equipe proposta originalmente no arquivo do Projeto encaminhado em
27 de março de 2013 a ANEEL teve alteração, pois alguns dos pesquisadores que
haviam sido selecionados para compor a equipe do projeto receberam propostas
de trabalho, portanto, houve a necessidade de uma nova seleção, mas somente
após a assinatura do contrato foi que a COPPETEC autorizou a contratação dos
demais integrantes para compor a equipe a qual contava somente com o
coordenador Ramon Costa e a pesquisadora Julia Campana. Devido ao novo
cenário conforme supracitado foi aberto um processo seletivo, coordenado pelo
Prof. Ramon Costa, mas que devido à dificuldade e escassez de disponibilidade
de interessados e, visando não comprometer o andamento e quiçá o sucesso da
dos estudos da pesquisa do referido projeto, o coordenador Ramon abriu o convite
para alguns alunos formados do curso de Engenharia e Controle e Automação da
Poli/UFRJ os quais haviam se inscrito para o mestrado no Programa de
Engenharia Elétrica da COPPE, sendo selecionados os pesquisadores para o
projeto: Renan Salles de Freitas, Gabriel Alcântara Costa Silva e Eduardo Elael
de Mello.

Posteriormente, durante a execução do projeto foi detectado a necessidade de um
suporte para a demanda burocrática do processo administrativo para o projeto que
até então não estava no escopo da proposta inicial encaminhada a ANEEL.
Portanto, um outro processo seletivo foi realizado para preencher o cargo de

Assistente Administrativo sendo selecionada Alana Monteiro com admissão em 03
de março de 2013, respeitando-se o regimento interno da Fundação para seleção.

Importante deixar registrado que no âmbito dos projetos de Pesquisa e
Desenvolvimento, a Fundação COPPETEC pode fornecer um suporte ao projeto
para seleção de pessoal, entretanto, o critério para a seleção e a escolha da
equipe, segundo a política da Fundação, fica a cargo do coordenador do projeto.
As restrições são analisadas pela Comissão de Ética, que recomenda uma
composição da equipe, responsável pelo desenvolvimento do projeto
predominantemente envolvida com a Universidade, ou seja, é composta por
professores, alunos e técnicos da Universidade.

Segue alguns dos critérios que são adotados pelo regimento interno da Fundação
para a seleção:

Seleção de aluno de Iniciação Cientifica
\begin{itemize}
  \item Estar regulamente matriculado;
  \item Ter boa avalição acadêmica;
  \item Ter experiência (caso seja necessário para área de atuação) e;
  \item Ter noções de língua inglesa.
 \end{itemize}
  
Seleção de aluno Mestrado
\begin{itemize}
  \item Estar regulamente matriculado em um dos programas de pós-gradução;
  \item Ter boa avaliação acadêmica
  \item Ter possibilidade de alinhar seu tema na linha de pesquisa do projeto
contratante;
  \item Ter experiência com projetos de Pesquisa e Desenvolvimento e;
  \item Ter inglês fluente.
\end{itemize}

Seleção de Pesquisador 
\begin{itemize}
  \item Ter ensino superior reconhecido pelo MEC;
  \item Estar regularmente matriculado em um dos programas de pós-graduação;
  \item Ter experiência profissional na área do projeto;
  \item Dominar as ferramentas de trabalho (dependendo do escopo do projeto, as
habilidades dos candidatos podem ser avaliadas para garantir que possuam o
conhecimento necessário para a execução do projeto) e;
  \item Inglês fluente. 
\end{itemize} 

Seleção de Assistente Administrativo
\begin{itemize}
  \item Ter diploma de técnico em administração;
  \item Estar cursando Graduação em Administração;
  \item Ter experiência na área de Administração
  \item Dominar as ferramentas pertinentes ao proposto no escopo do projeto (as
habilidades dos candidatos podem ser avaliadas para garantir que possuam o
conhecimento necessário para a execução do projeto) e;
  \item Ter noções de inglês.
\end{itemize}

Aproveitamos todo esse contexto para mostrar que assim como todo o projeto tem
suas dificuldades em recrutar e até mesmo manter um recurso humano, no projeto
de P\&D não é muito diferente, enfrentamos todas as dificuldades principalmente
pela dificuldade em manter um pesquisador frente as várias ofertas de trabalhos
que os pesquisadores recebem de empresas''.

\includepdf{PDFs/Tabela_Gastos.pdf}